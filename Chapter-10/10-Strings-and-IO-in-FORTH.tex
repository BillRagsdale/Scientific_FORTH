Wronm

MIMMI/OhFORfl-t 249

Strings and 1/0 in FORTH

contents
§1 Standard I/O in FORTH 249
52 Standard l/O in Scientific FORTH 250
§§1 Disk file input 251
§§2 Disk file output 253
53 Logging screen activity 254
§§1 Redirection by re-vectoring 254
552 Redirection using MS-DOS resources 255
§§3 Redirection while DEBUGging 257

This chapter describes some input and output (l/O) functions

and string handling facilities needed for scientific computing.
We will illustrate with words that capture screen output to a disk
file, and with a word that translates a FORTRAN expression
—entered from keyboard or read in from a file- to FORTH
code.

51 Standard I/O In FORTH

One area where more conventional languages excel FORTH in

convenience for the casual technical user/programmer is
standardized I/O. FORTRAN offers all the needed functionality
—number formatting, strings, etc.— albeit in a somewhat rigid form.

BASIC, a more modern language, offers even more convenient
built-in I/O facilities than FORTRAN.

I confess I personally do not find the practices of C or Pascal
congenial, but most programmers seem able to live with them.

FORTH possesses no standardized I/O words except for writing
to the screen and inputting from the keyboard. Thus there are no

omvmim-umw.

250

§2 Standard l/O in Scientific FORTH

ChlptoHO—SWIMIIOMFORTH SclemlflcFOfl

required words for redirecting output to printer, disk file
plotter.

FORTH lacks standard I/O functions for several reasons:

   
     
 
   

o Historically, FORTH incorporated its own (rather primitiv
operating system (OS) on many machines. The aim was co
pactness rather than so histication, so it eschewed built-in
or BIOS ("Basic Input] utput System) functions and/or devi

drivers. i

e FORTH has been implemented on so many machines tha
hardware independence in [/0 has seemed an unattainab
goal. ‘

FORTH developers have therefore deemed it better to let eaci
programmer develop his own favorite techniques and specializes
device drivers. The result is Bedlam.

While the freedom to custom tailor U0 is very nice when

unusual problems crop up, no one likes having to roll his own
subroutines to do routine tasks (95% of all U0 is routine). .5-
scientific FORTH dialect needs to standardize this routine 95 9‘:
This book is intended for dialects like HS/FORTH that run under
an operating system1 and are file oriented. Such dialects new;
sarily contain a reasonable spectrum of words for disk, screen an
printer I/O, equivalent to those in HS/FORTH that I will emplo:
in this chapter.

Some FORTHS are screen oriented. That is, they are designed ts
read and write blocks of 1024 bytes, displayed as 16 lines of 64
characers per line. This was a great idea for primitive machineii
like the Timex/Sinclair ZX81 or Jupiter Ace, but in my opiniod
is thoroughly inappropriate for modern work stations with sop '
ticated displays.

 

1.

Such as MS-DOS, OS/Z, Pick, UniX. etc.

hammers: thmwnroen-i 251

I strongly recommend users of screen oriented FORTHs to up-
date to file oriented systems. Failing that, they should consult any
of the numerous introductory books, or articles in journals such
as Dr Dobb ’3 Journal, for suggestions on using 1K blocks to store
and input numbers and text.

“1 Disk fl. I‘tpm
We now tackle the problem of inputting data from disk files.
HS/FORTH provides the words OPEN-INPUT, CLOSE-
INPUT, < FILE, 8", G#, GS, NR, and $-> F that will suffice for
our needs . Their effects are:

OPEN-INPUT \ open a file for input, whose name is a
\ string whose address is on the stack.
CLOSE-INPUT \ close an open input file

< FILE \ get input from the open tile

$' \ put a string at PAD (terminate with ')

G# \ get a 16-bit number from open device

GS \ get a counted string from open device
\ (will not read past a CR-LF)

NR \ skip CR-LF to next record

3- > F \ convert the string at PAD to a floating
\ point number on the 87stack.

Me: Most operating systems handle sequential files by main-
taining a pointer into the file. After a datum is read, the pointer
is advanced. Suppose the file datfil has 3 numbers in it: 17, 32
and -8. Then the phrase

5' dat.fil' < FILE G#

will place 17 on the stack. A repetition will place the number
32 on the stack, and a third repetition will place -8 on the stack.

 

2. Voltaireonccrcmarkedthatifsuandidnotcx'utitwouldbenecessaryforGodtocreatehime
yourFORTl-lsystemlachwordswiththesefmaimgbyallmeans,inwntthemAMS-DOS
programmer's manual (see below) isvital forthis task.

OJdianVNobiotm—Almmerved.

ciuptutommuomeonm Sclanflk

     
    
 

We shall use the following data file convention for arrays:

e The first number is the type of the array.
0 The second is the order N.

o the third is the number of entries— N2 for a
(ZARRAY), Nfor avector (1ARRAY).

a (fincfitln. more ASCII blanks (20h, 32d) separate the numbers
I e e

o The entries in a ZARRAY are stored rowwise.

NM..- . .vnh.__..~—m-

These conventions let one word serve for 1ARRAYs and ZARRA

Coding from the top down, we plan to enter the following phras;
to load the file:

IN SUPERSEG 100 LONG REAL 1ARRAY A{
A{ s' datafll“ FILL< FILE

To minimize our labor let us plan to have the word FILL < FILE ;
close datafil after reading it in. For safety we also should Cheri“ .
whether the type and order of the array we are filling, and that (a i
the array stored in the file are commensurate.

To check whether the file is too large or ofthewrong type, we have it

0 VAR FILE.TYPE

O VAR FILE.LEN

: READ.FILE.STATS (FILE
G# IS FILEJYPE
G# IS FILE.LEN ;

: MAT> =FILE? (
DUP D.TYPE OVER D.LEN (
FILE.LEN > (
SWAP FILEJYPE = (
AND NOT
ABORT‘ File bigger than array' : i

 

The rest of the definition will look something like this:

:COMPLEX? FILETYPE 10>; \isitcompierfl
:EOR? <FlLEGSCOUNTO-;

: FANR EOR?
IF <FILE NRGSDROPTHEN \startnewrecord
PAD 5- > F DROP ; \ convert to floating
:FILL<FILE (A($fiiespec--)
OPEN-INPUT open file for '
MAT>-FILE? \checktypealength
FILE.LEN O
DODUPIO} (--A{segofltype)
COMPLEX?

lF F#.NR THEN \ enter 2 #‘s

F#.NR GI \puteway
LOOP DROP

CLOSE-INPUT ; \ clean up

552 oiskmooutput

Suppose one wants to write an array to a file. We can use

A{{ s- datafil" MAT> FILE

01’

e{ s' data.fil' vec > FILE

where we define4

:MAT>FILE (a{{ $adr--)
MAKE-OUTPUT \$adr = tile.spec
>F|LE DUP D.TYPE .

DUP D.LEN DUP . “2 .
.M CLOSE-OUTPUT;

 

4.

WeuseMandes-omaiapteri'l‘hewords >HLEMAKE-OUTPUTandCLOSEOUTPUT
aretheouputmlqueaoftheinputwords <FILE,etc.

thlUtVNoblaim—Almmuvad.

Chapterto—smngundI/omrom SclentlflcFOR

 
     
  
 
 

:VEC>FILE (V( $tilespec--)
MAKE-OUT PUT
>F|LE D.TYPE .
D.LEN DUP . .
.V CLOSE-OUTPUT ;

Logging screen activity
Most FORTHs handle output to a device by vectoring thr 7
a user variable called EMIT. This allows relatively easy on '
put redirection. Hence, for example, everything sent to the scr
can be echoed to the printer (in an MS-DOS system operating
a PC-clone this is trivial: just press ctrl-PrtSc to toggle printefl,
logging on or off). i

The terminal sessions reproduced in preceding chapters were
captured by logging development sessions to disk files. This is far
safer than logging to a memory segment or ram-disk because it
produces a record that remains even if one’s experiments crash
the system. Logging to the printer is also permanent, but the
results are not easily inserted into a book!

HS/FOKI'H permits (at least!) two simple methods for redirect: .
ing output, which we now describe.

1'
l

§§1 Redirection by re-vectoring
The first word in this set of definitions creates the logging file.

: MAKE.LOG.F|LE
CR ." Insert a formatted disk in drive A:'
CR .' Press any key to continue, Esc to abort ...'
KEY 27 = ABORT" ABORTED!‘
$" AzLOGJXT' MAKE-OUTPUT ;

: REOPEN.LOG.FILE
$' AILOGTXT" OPEN-OUTPUT;

:FWEMIT \awordtorepiacastandard EMIT
DUP \duplicetacharacter
FEMIT \emittoopenlile
OUT 1-I \decrementtheoutputcount
WEMIT; \emittotheCRT(thisincrementsOUT)

:FCRT CFA' FWEMIT IS EMIT ; \vectornewroutine

: LOG-ON FCRT \ all I/O to tile 8- CRT
IO-STAY ; \ normally FCRT would stop
\ after word executes.
\ lO-STAY leaves FCRT on.

:LOG-OFF CLOSE-OUTPUT IO-RESET ;
\ IO-RESEI’ restores I/O
\to standard device

Redirection uslng MS-oos resources

Aisembly language programmers familiar with MS-DOSS ver-
ions 2.x and higher will be aware that the operating system
includes many useful functions. Those employed here are in-
voked by means of an “interrupt" (meaning the CPU halts what
it was doing and performs the system function).

The interrupt code to invoke functions is 21h (hex). The system
functions are usually specified by placing the function number in
register AH, and other needed information is placed in other
registers. Then an INT 21 instruction is issued.

For example, to create a new file (or to overwrite an old one of
the same name), a program operating under MS-DOS would
include the following code:

MOV DX, FILESPEC ; FILESPEC is address of
; a string ending with ASCII 0
; containing [d:][\path\]file.ext

MOV AH, SCH ; the function number of 'create'
MOV CX, OH ; the file attribute
INT 21 H ; now perform the interrupt

 

S.

O Mia'osoft Corporation, 1983-.

256 Chapter IO-Strlnge and I/O In FORTH Sclendflc

  
    
   
  
   
  
    

Consulting the MS-DOS Programmer's Reference‘s reveals
Function 3Ch either leaves an error code in AX (if there was
error - depicted by having the carry flag bit CF = 1); or Ie
the file handle number (if all went well - depicted by CF = 0) in

The HS/FOKI‘H word MKFILE expects the address of a counte
string on the stack. This string contains the drive, path an
filename of the new file. Then MKFILE performs the inte
(using code akin to that above) to invoke the function, issues a
error message if something went wrong, and leaves the file hand
number on the stack if all went well.

This file handle number is MS-DOS’s way to refer to the new fill
internally. All subsequent references — say to read or write — ar
made to this number rather than to the FILESPEC (which wo _
be much more tedious). The word MAKE-OUTPUT used about:
contains MKFILE as a component.

Now what does all this have to do with the price of bananas in the.
Maldive Islands? One of the features of MS-DOS is to treat
everything like a file, including the printer, CRT; serial ports, etc.
Thus output intended for the printer can be sent to a file by
fooling DOS into thinking the file was the printer. This can be ‘
done by placing the file’s file handle number where the printer’s
should be.

But MS-DOS actually has functions we can use to splice two file
handle numbers so that they are subsequently treated as a single
file. These are Function 45h (duplicate a file handle specified in
BX: take an already open file and return a new handle that refers
to the same file at the same position) and Function 46h (force a i
duplicate of a file handle specified in BX: take an already open
file and force another handle in CX to refer to the same file at
the same position; if the file handle specified in CX was open
close it).

 

Function 45h is used in the HS/FORI'H word DUPH, and 46h is[
used in SPLICEH. We can combine them with the (already-

 

6. Radio Shack Cat. No. 26-5403.

meant" custom-remnant» 257

defined) printer-logging word PORT, as well as with IO-ITAY
and CRT, to let DOS perform the redirection of output:

0 VAR LOG-HANDLE
o VAR em VARs for storing handes
: LOG-ON 3' AzLOG. MKFILE
\ make his, leave handler!
IS LOG-HANDLE \ store file handles
4 DUPH IS P-H# \ make new printer handie#

\ and store it
LOG-HANDLE
4 SPLICEH \ splice the handle#'s
PCRT lO-STAY ; \ redirect output to
\printer&CRT&leaveon
: LOG-OFF P-H# DUP
4 SPUCEH \ splice printer back to primer

LOG-HANDLE CLOSEH
\ flush to log-file, close
CLOSEH \ close new printer handle
CRT ; \ go back to CRT only
BEHEAD" LOG-HANDLE P-H# \ make VARs local

“3 Redirection while DEBUGging

The only problem with the logging functions defined above is

that they depend on an idiosyncracy of MS-DOS called the
environment. That is, when we execute a program from within
another using the HS/FORTH word 005' we automatically
create a new shell and run all DOS commands under a second
copy of the command processor, COMMANDCOM . This
means the environment variables —specifically the handles of
open files — will not be valid. Hence the output sent to the screen

by a program being run via DOS“ will not be echoed to the log
file.

 

7. DOS'muubedefinedhmachinehnguageminvoketheappropriaeDOShmbnmrough
anotherINT‘Zlflall

258

Chapter 104W endl/OInFORTH Sclentlflc

    
   
 

Tb capture screen output while DEBUGging, I resort to a utili
published in PC Magazine called PRNZFILE . This utili .
replaces the DOS printer-interrupt routine (INT 17h) by
modification that sends the output to a file instead of the printer
The assembly language listing can be downloaded from PCMag.
Net, and is included on the program disk as a convenience to the l
reader. ‘
To use this utility from within PIS/FORTH, first say 008' < cr >
to get into the DOS shell. A DOS prompt such as C> should
appear. Now give the DOS commands

C > prn2file Iog.fil [/b8]

C > ctrl-PrtSc (toggle echoing to printer) ,

C > debug ( get into DEBUG) -‘

...... debugging session ......

-q (quit DEBUG)

C > ctrl-PrtSc ( stop echoing to printer)

C > prn2file (with no argument, closes Iog.til)
(and redirects back to printer)

C >type Iog.fi| ( check that it has worked)

C > exit ( get back to HS/FORTH)

Although this is somewhat cumbersome, it does what I need, and
I did not have to write my own.

With the illustration of how to use I-IS/FORTH's MS-DOS exten«
sions to fill an array from a file, to send the contents of an array
to a file of standard structure, and to capture programming
sessions to a file, we have reached the end of the U0 trail.

 

8.

PRNZFILE. 1.0 O 1987 74'“ Communications Co. Written by Tom Kihlken.

