WM”! M7-WMhFOfiTl-t

Complex arithmetic in FORTH

Comm.

§1 The complex number system 144
§§1 Cartesian representation of complex numbers 144
552 Polar representation of complex numbers 147

52 Load. store, manipulate tstack 148

53 Arithmetic operations 149

54 Roots of complex numbers 151
§§1 Complex square roots 152
§§2 The complex square root program 154

§5 Complex exponentials and trigonometric functions 154

56 Logarithms 155

143

ne of the most crucial features of FORTRAN for scientific
computation is the ease with which it embeds complex arith-

metic into formulae. No other compiled language has this fea-
turel. From time to time someone gets a bright idea, and adds
complex arithmetic to Pascal via subroutines (since Pascal func-
tions can return only a sir¥le number, and complex functions
must return two numbers) . The same problem would afflict C
and the new structured BASICS.

Recently, a mechanism for adding complex arithmetic to Pacal
by defining a stack and using postfix notation has been proposed3.
Does this sound at all familiar?

This chapter deals with complex arithmetic and its implementa-
tion in FORI'H. We do not consider complex numbers whose real

 

1- APL yacefully admits complex numbers and functions, but is an interpreted language.

1 Dr). Clark, “Simple Calallations with Complex Numbers",Dr. Dobb': Journal, October 1984. p. 30.
3.

D. Gedeon, “Complex Math in Pascal”. Byte Magazine, July 1987. p.121.

eJumVNounaaz—ungmm.

Chapter 7 - Complex arlthmetle In FORTH Scientific FORTH

and imaginary parts are integers, since these are virtually useless.
in scientific computing. Therefore we perform complex arith-
metic solely in single— or double-precision floating point format.
We begin with a brief review of the complex number system.

The complex number system
he algebraic equation
x2-1=0 (n

has 2 roots, :1, in the standard number system (real numbers).
But the (otherwise quite similar) equation

f+1=0 (»

has no roots. That is, there is no ordinary number which, whet
squared, gives a negative result. Eighteenth century mathemat-
icians misliked a state of affairs wherein some polynomial equa-
tions of n’th degree had n roots, whereas others had n-2, n-4, etc-
This seemed disorderly and unpredictable. How much simple.
life would be if every polynomial of n’th Order could always b1
factored into n primitives of the form (xll are roots)

ao+alx+a2x2+m+a.,x'l

(3)

E an(x — x1)(x — x2) (x — x“)

§§1 Cartesian representation of complex numbers

In order that every polynomial have n roots it was necessary tn
extend the idea of number to include objects of the form

z=xl+yi (4)

where i is —by definition— a “number" whose square is —1; 1 it
a “number” whose square is + 1, and x and y are ordinary num!
bers. Conventionally, x is called the real part of z, and y thl
imaginary part:

I = R£(Z) . y = Im(2)- (5) !

amr-cmwmmomn 145

Thus the solutions of the polynomial equation
21 + 1 - o (6)
are
2 - 01 3: n (7)
(that is,x=0,y= :1).
Tb reassure readers who find uncomfortable the notion of a

“number” i = FT whose square is negative, it is possible to find
a 2X2 matrix representation for 1 (unit matrix) and l :

10 01
1=(..).:=(-..)

It is easy to verify that —in the sense of matrix multiplication —
ixi=—1,1xi=ixl=l,andlx1=1. (9)

These are just the usual multiplication rules for complex num-
bers. Note that x l and y i then mean multiplication of a matrix
by a scalar:

“(33), mil 3)

z=x+iy5 (3y i)

The complex numbers obey all the algebraic rules of ordinary

arithmetic — commutative and associative laws of multiplica-
tion and addition. They are complete in the sense that when we
multiply or add two complex numbers we get a complex number,
not some other kind of number:

(a+ib)(x+iy)=(ax—by) +i(ay+hx) (10)

OMVNoualflz-Almnaarvad.

146

4.

Chapter 7 - Cornflex arlthrnetlc In FORTH Selenflflc FORTH!

(a+ib)(x+iy)=(ax-by) +i(ay+bx) (10)

For every complex number, 2 =x + r‘y, there is a corresponding.

complex conjugate complex number,
2' = x - iy. (11)

Each non-zero complex numberz has a multiplicative inverse 1/z.,

To see this, multiply numerator and denominator of NZ by thd .

complex conjugate Eq. 11, and note that

22' Ez'z=x2+y2

is a non-zero real number. We can therefore calculate its inversc‘

‘_4

A

 

by ordinary division, and (scalar-) multiply 2 ° by the result:

1 z‘_ x _y
— 2,:[22 —1222]. (12)

 

22 2222 x; ‘7’? X2 +.Y2

To divide one complex number 21 by another, 22, invert 22 an:
multiply: 21/22 E 21 X (1/22), i.e.

 

 

Z
12(x1+iyl) x 2‘2 2 -t' 2" 2 (13)
22 x2 +)’2 13: +Y2

(Numbers that satisfy the addition, multiplication and closure
properties of real or complex numbers are said to form a field.)

It is easy to prove that evenwhen the coefficients of an n’th degrc l
polynomial are complex, it has n roots that can be represented a;
complex numbers.

Complex numbers can be used to represent points in the x-y plant:
with y plotted vertically and x horizontally‘. ( Sometimes thi
graphical representation is called an Argand plot or Argantl
diagram. The x-y plane is then called the Argand plane.)

J

L

The 2vdimensional graphical representation of complex numbers makes complex arithmetic at-
tractive for computer graphics.

 

{meant-n

 

mv-Wmmronm 147

 

Im(z) z

 

 

x Re(z)

 

 

 

F .7-1Representl acom numberasapolntlntheArgand
lg plane, In Gangeslan apnlgx polar coordinates

§§2 Polar representation of complex numbers

The complex number 2 = x + iy is said (remember the Argand

plane) to be represented in Cartesian form (after Rene Des-
cartes, the inventor of analytic geometry, Le. the idea of graphing
equations). However, it is equally valid to use the relationship
(Euler’s theorem)

e'05c0s0+ isin0 (14)
to write the polar representation of z,

z=nr , r20. (15)

Clearly, since 2' 5 M4), we have

NI—

r = J? = (x2 + yz) (16)
ands
9 = tan—1(y/x). (17)

 

See Ch. 3 “"1 for definitions of the inverse trigonometric functions.

OJuthNobletm-Alnohtareaewed.

148

Chapter 7 — Complex arhhmetlc In FORTH Scientific FORTF

Note that 0, sometimes called Arg(z), is defined only up ta
multiples of 2:! — that is, adding 23 to an angle changes neither
sine nor cosine because of their periodicity. The relations between
(try) and (r, 0) are illustrated in Fig. 7- 1 above: '

52 Load, store, manipulate fstack

We now define FORTH words to perform complex arithmetc.’

All complex operations will be prefixed by the letter X6. We
assume all complex operations are performed on the FPU ice
speed, hence we shall give 87stack diagrams.

In 87stack (or fstack) diagrams 2 stands for complex number, i
for real part, and y for imaginary part. Where useful, the 87stacl
diagrams show the operations decomposed into real and imagii
nary parts. By convention the imaginary part is higher on the
fstack than the real part.

By now most FORTH code should be self-explanatory. Mair.
words have been coded in high-level FORTH because the over
head of threading is negligible compared with the time spent
executing.

With by-now obvious meanings we have

\ - - single 81 double-precision complex fetch and store.
: X@ DUP 832@ 4+ 832@: (adr-- 87:--z)

: X! DUP 4+ R32! R32! ; (adr-- 87zz--)

CODE 8+ BX 08 IW ADD. END-CODE

:DX@ DUP R64@ 8+ RS4@; (adr--87:--z)

:DXl DUP 8+ RS4| R64l; (adr-- 87:z--)
\ - - and complex fetch and store.

 

The letter C is more mnemonic, but FORTH conventionally reserves C for prefiaing byte (“ch82
acter”) operations — as in C@, Cl, etc. The complex extension supplied with ITS/FORTH Ilsa!
theprefiwahenit isunambiguous (asinC+,C—.C',C/,etc.),andCPwhenCalonewon‘t
do, as in CP@, etc.

WFOMH MLWMMFOHTH

\Madt

:REAL FD OP;
:IMAG FPLUCK;
:CONJG FNEGATE ;

:XDROP FDROP FDROP ;
:XSWAP F4R F4R ;
:XDUP FOVER FOVER

93 Arlthmatle operations

149
(87zz--x)
(87:z--y)
(87:xy--x,-y)
(87zz--)
(87:2122--z2z1)
(87zz--zz)

be standard complex operations should be virtually self-ex-
planatory. For efficiency we define multiplication and division
of a complex number by a scalar, as well as complexx complex

and complex/complex.

:CMPLX F=O ; (e7:

x--x0)

: X+ FROT F+ F-ROT F+ FSWAP;

: x- FROT FR- F-FlOT F-

: XOVER FSP F3P; (87:
: XtF (87:
FUNDER F* ( 87:

F-ROT F‘ FSWAP -

: Fax FROT XtF ; (87:

FSWAP;
2122-- Z122 21)
xya --ax ay)
--a x bx)

x ab--ax bx)

\CODE XIF 2FMUL'. 1FMUU’. END-CODE

: th FNEGATE FSWAP ;
: X/F 1/F XtF ;

The critical operation of complexx complex can be defined in

high level FORTH for portability:

: X'
XOVER X‘I FROT

F3X FSWAP
F4X FSWAP
F'X

X+ ;

(87: 2122 --z1"22)

(87:--abx -bay)

X'F (87:--abx—by ay)
(87:--a ayx b -by)
(87:-- —by ayx a b)
(87:-- -by ay xaxb)

Actually, complex multiplication is sufficiently involved to be
worth defining in code. Here is a code definition for the 801187
family of FPUs. The equivalent for the Motorola 68881/2 family
is virtually identical, allowing for minor differences in Intel and

omvuouhtm-Mmm.

150 Chaptar7-CanplaxarltlnatlehFORfl-l wrath

 

Motorola assembler mnemonics, as well as for the fact that 1%
68881f2 has registers but no stack.
\oneration 31mm
CODE X' \x y a b
3 FLD. \x y a b x
2 FMUL. \x y a b 1:3
4 FXCH. \xa y a b x
1FMUL. \xayabxb g
1 FXCH. \xa y a xb b ;_
3FMUL. \xayabxyb ;
4 FSUBRP. \ xa-yb y a bx
2 FXCH. \xa-yb bx a y
1 FMULP. \xa-yb bx ay -
1 FADDP. \xa-yb bx+ay
ENDCODE (xyab--xa—ybxb+ya)t

Once we have multiplication, division is easy:

: XMODSQ (87: xy-- x“2+y“2)
F“2 FSWAP F“2 F+ ;
: 1/X CONJG XDUP XMODSQ
FDUP FO= ABORT' Can'tdividebyO‘ X/F ;
:X/ 1/X X’ ; (87:2122--21/22)

With the preceding discussion and referring to Fig. 7-1 at
page 147 the FORTH words to accomplish Cartesian-polar" trans
formation and vice versa should also be fairly transparent :

: ARG FSWAP FPATAN ; (87: x y - - atan[y/X])
SYNONYM FATANZ ARG
\ FORTRAN defines FATANZ so we do also.

:XABS (B7:z--|z|)
XMODSQ FSQRT ;

: >POLAR (87:xy--r0[radians])
XDUP XABS F-ROT ARG ;

:POLAR> (87:r0[radians]--xy) 5

FSINCOS FROT X'F ;

7. SYNONYIlisanHS/FORTHinnovationtonveaomecodebyavoidinngATm no:

mam anathema-mama 181

'4 Roots of complex manners
n'th root of a complex number 2 is that complex number,
2“, whose n'th power is the original number. That is, as for

realnumbers,
(1%).I Hz (18)

it might seem almost trivial to define the n’th root of a complex
number in polar representation:

z-re” (19)
hence
2 VII 3 r Vile i9/n (20)

Certainly we know what we mean by the n‘th root of a positive
real number, and from Euler’s theorem we know how to evaluate

e"""'=-=cos9+t'sin2 (21)
I: II

However, we can also think of the n'th root 3 a solution of the
polynomial equation in the variable to

w" — z = 0. (22)
The lbndamental theorem ofalgebra proves that a polynomial of
n'th degree has exactly 11 roots; this implies there are n distinct

values of w that satisfy Eq. 22; ie., there are n distinct roots of 2.
How can we generate them? Let us call the root shown above

(in Eq- 20)
we I: r 1/n e iO/n
Then we can multiply wo by a factor

x.=e”"*’",k=o, l,...,n-1 (23)

to obtain it different numbers,

OJtIU'IVNoflatm-Alrldaaraaarvad

152

Chapter7-ComplexarlthmetlclnFORTl-l SclentlflcFOFlTh]
Wk=Woxk, k=0,...,n-1 '
the n’th power of each of which is z. Clearly, if k increases pas

n-1, the numbers wk simply repeat . Since it is neither possible:
nor desirable for a complex function to return all n roots of 2, we!
choose the principal one, wo. All complex root- finding functions:
should obey this convention. In general the simplest algorithm tat
calculate the n’th root of a positive real number is

r U" = eln(r)/n ' (24)

Then to complete the job of evaluating wo we would need to
calculate a sine and a cosine. Thus, 3 divisions, 2 multiplication;-
and 4 transcendental function calls are generally needed to eval-
uate the n'th root of a complex number.

However, for square roots (n = 2) there is a much more efficient
method based on ordinary square roots, which we shall now
describe.

§§1 Complex square roots

Our phase convention means the phase 0 of the square root of 2:
must lie between 0 and 7:, since that of z lies between 0 and 27!.

The algorithm can be understood using the half-angle formulae:
for sines and cosines (we used these to develop the trigonometric
functions in Ch. 4 §6):

1

cos (g) = (“Li-5A6) 2 (258)
1

sin (g) = (—1‘;°s 9): (25b)

 

8.

Recalleoiez'” - 1.

amt-Gammmronm 153

Nowweusethefactthatifz -x + r'y,andifw - a + ibisits
square root, then their polar representations are

x r cos 9
(y) - (nine) (26")
a _ V— cos(0/2)

(b) («E sin(9/2) ) (26")

Therefore the principal root is
1

a mm) [g0 +x):|2, (272»

= [awn]? (27b)

That is, as we can easily see from the Argand diagram, Fig. 7-2
below, if Im(z) is negative, then Re(w) will be negative, and vice
versa.

 

 

Im (z)

Re(z)

 

 

 

 

.7-ZCompIaxnunbeIanthlaalnalyMandlts
Fig WWW“

onlanltNoblalflz—Almreaarvad.

154 ChapterT—ComplexarlthmetlclnFORTH summon}

§§2 The complex square root program =
We now translate these equations into high- -level FORTH witlr:

comments:

: XSORT ( 87: z - - z“1/2)
FSWAP XDUP XABS (87: -- yxr)
FDUP F0= \retumO if |z| = q
lF FDROP XDROP X=O EXIT THEN
XDUP F+ (87:--yxrr+xr
F2X F- (87:-- yr+x r-x)
F2/ FSORT (87:-- yr+x b)
F2X (87:--br+x)

F0 < \ get sign of |m(z)
F2] FSQRT (--f87:--b |a|L
IF FNEGATE THEN \fix signofRe(z“1/2)
FSWAP ;

We test to make sure I: | > 0, since we do not want the possibility
that |z| -x or |z| + x works out to be negative (albeit small!
through roundoff, thereby generating an error in the square root
routine.

§5 Complex exponentials and trigonometric functions

Complex exponentials and trigonometric functions are nearly
self-explanatory. They are based on Euler’s theorem,

ew=c050+isin6.
Thus,

ezEe x+iY=ex 39': e’(cosy+isiny) (28)
Similarly,

-1 ix-y y-it

Whip +. ) (29a)

- -1 ix-y_ y-ir

srnz — E (e e ). (29b)

Hence the code for the complex trigonometric functions is

oJuunVNoblelm—Alrlghtareeerved.

: FSINCOS (87: x - - cos[x sin[x] )

F2/ FTAN (87: -- ass-tan 11/21)
FDUP F"2 (87: -- aa“2)

F-1 XDUP F+ (87: -- ea“2 1 1+a“2)
Fax F- (87: -- a 1+a“2 1-a“2)
FOVER F/ (87: -- a 1+a“2 cos[x))

F-ROT F/ F2' ; (87: x - - cos[x] sln[x] )
\ note FSINCOS ts mlcrocoded on the 80387

:XEXP (872x y--e“x'cos[y] e“x"sin[y])
FSINCOS FROT FEXP X'F ,

: X2] F2/ FSWAP F2] FSWAP ;

: XSIN (87. x y - - sin[x]cosh[y] cos[x]sinh[y])
FNEGATE FEXP FSWAP FSINCOS F‘ X
XDUP 1/X X- X2/ FSWAP FNEGATE ,

: XCOS (87: x y - - cos[x]cosh[y] -sin[x]sinh[y] )
FNEGATE FEXP FSWAP FSINCOS FX
XDUP 1/X X+ X2/ ;

§6 Logarithms
The logarithm of a complex number must be defined by the
polar representation of the number. Thus, using the fact that

log,(ab) = log¢(a) + logc(b) , (30)
and that
log.(ez) = z , (31)

we find it consistent to define the complex logarithm as
log.(z) =1og,(re‘”) s log,(r) + i0. (32)
Thus,

:XLOG (87: x y - - |n[r] atanly/Xl)
>POLAR FSWAP FLN FSWAP ;

This completes our dissertation on complex arithmetic.

cmvuouotm-Alugmm.

Chanel’T—CmpiexarlhmetlclnFORTH

cluanvuouotm-Almm.

SclendflcFOflTH

