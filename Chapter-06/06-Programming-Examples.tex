WFORTH awe—WW

Trogramming Examples

n

Sentente

Chapter 6 - Progremmlng Examples
51 Infinite series
§§1 Examples of infinite series
§§2 Numerical examples of convergent series
§§3 The infinite sum program
§2 Transcendental equations
5551 Binary search
§§2 Regula falsi
53 Ordinary differential equations
§§1 Runge—Kutta method
§§2 An implicit Runge-Kutta formula

115

116
116

118
122
127
127
128
131
132
136

115

116 Ctuptere-PrograrmiingEx-mplee ScknflflcFOR‘n

This chapter illustrates how we may apply the floating poin
extensions of FORTH, developed in preceding chapters, t4
some standard problems in numerical analysis.

51 Infinite series

Frequently we must evaluate a function defined by an infinil
sum

ID) = 2 Cut" (1)

n=0
where x is a real number and c. is an infinite sequence 0
coefficients. The extensive mathematical theory1 of such fun:
tions can be summarized as follows: the series of terms only ha
meaning if —for fixedx — the sequence of partial sums

N—l
fn(x) = E cnxn (2)

n=0

has a definite limit.

§§1 Examples of Infinite series
erhaps the formal definitions will seem clearer after sorn
concrete examples. We now examine some divergent an
convergent series.

§§1—1 Divergent examples
What does it mean to say a series diverges? Consider the seric
whose terms are all 1's (that is, c,I = 1, x= 1):

1+1+1+1+... (3)

The partial sum fN is just N, and therefore increases without lint
as more terms are added. The series diverges.

 

1. The Handbook of Mathematical Function: ed. Milton Abramowitz and Irene Stegun (Dover H
lications, Inc, New York, 1965) — henceforth abbreviated HMF- is a mine of useful informa-
tion and references, on this as well as many other aspects of numerical analysis.

cums-Wm 117

A harder case is the series whose terms are alternately 1's and
-1's:

1—1+1-1+1—... (4)

Depending on how the terms are grouped together, the partial
sums can have any value. Certainly the partial sums do not settle
down to any definite value. The series diverges.

Yet a third case is the harmonic series

111.1

fN =1+2+3+- +. .+N. (5)

It can be shown that for large N, f" is approximately log¢(N), so
the harmonic series diverges.

§§1—2 A convergent example
Are there ever mes of infinite series that do mean something?
Obviously, or there could hardly be a theory of them! An ex-
ample2 is c. = 1 , x = V: . Here the partial sums

fo=1

1
f1-1+2
1 1
fz=1+§+3 (6)
mnfN E but 1—(1/22”_2
N-om N-om 1_V2

do have a definite limit, so the series converges.

 

Thisistheenmpleofho'sparadaswhereyouareatoneendofasofaandanattractiveper-
soaoftheoppositegenderisattheotherendYoumovehalfthedistance, thenhalfthe
remainder. Cleaflyuinfinitenumherofmoves'uneceswytoachievethedesuedpmsimity
Maccadingtoleno, younewrgetthere. AceordingtoEq. 6,however,youdogetthereand
inalalhtinseJohootl

emvmru-Mmm.

116 Chapters—Programmes SclentlflcFORTi

552 Numerical examples of convergent series
How do we tell whether a series has converged? As a practice
matter, we keep adding temis to the partial sum until it rs
longer changes, within the desired precision. Sometimes this ca
involve a great many terms, even when the series converges.
There exists an extensive mathematical literature on testing in
the convergence of an infinite series. Mathematicians have ale
developed many tricks for accelerating the convergence of slowt
converging series3. Consulting the literature in difficult cases i
strongly recommended — it can save time galore!

As an heuristic exercise, let us write some simple programs in
evaluate partial sums and see how convergence works.

First we sum the terms 2'“ from Eq. 6. The flow diagram is sham
in Fig. 6-1 below.

 

SETUP: N=0 SUM=1
é BEGIN

?CONTINUE ( 10 more terms?)
WHILE

    

YES

 

® 10 o no
2*H = 2*‘/2
SUM = SUM + 2*H
n = n + 1

. LOOP

. REPEAT

 

 

Fig. 6-1 Cornputlng2rheherdway

 

3. llMF,Ch.3§6ff.

WM”! Owe-WM. 119

The corresponding program is

15 #PLACES I setEtoiSdigrts' '
:NEXTIERM \

(n--n+1:: sum2"-n--sum' 2" M)

F2! FUNDER F+ FSWAP 1+
:SEtUP FINIT F21 F=1 0
:EXHIBIT CR DUP .' n = '

ZSPACES

FOVER .'sum 2 ' F. ;
:?CON'I1NUE CR "Another 10terms?" 7YN ;
\WNexpectsa'y'or' n' iromthekeyooardand
\leaves- 1-(‘true' )if“ y' is pressed, Oii'n
:SUM SETUP

BEGIN EXHIBIT ?CONTINUE

WHILE 100 DO NEXTTERM LOOP

REPEAT ;

.
I

This is what a run looks like:

FLOAD TESTSUM Loading TESTSUM ok
SUM
n = 0 sum = 1.00000000000000
Another 10 terms? Y

= 10 sum = 1.919902343750000
Another 10 terms? Y
n = 20 sum =1.99999904632568
Another 10 terms? Y
n = 30 sum = 1.99999999906867
Another 10 terms? Y
n = 40 sum = 1.99999999999909
Another 10terms? N ok

The partial sums converge rapidly to 2 (which, as we saw in §1. 1.2
above, is the exact sum).

For a second example, let us sum a standard infinite series repre-
sentation for :rr/4: We note that the function tan 1(x) (that is,
' arctan(x) ) has the infinite series representation

 

i. M, cs 4 9.42.

1 20 Chapter 6 - Programming Enmpiee Sclean FORTH

3 7
tan‘l(x)=x—%+x?5-x7... (7)
Since a 45° right triangle has equal height and base, 5the tangent

of 45° is 1 (side opposite over side adjacent). That is ,

a = -i _ _ l 1 _ 1 l
4_tan (1)—1 3+5 7+9... (8)

Actually the series Eq. 8 is slowly converging and therefore a poor
way to compute 7r . But anyway, let us proceed. The flow diagram

is now that of Fig. 6-2 below.

 

SETUP: flag=1 n=1 sum=0 (--1n::--O)

BEGIN
EXHIBIT \ display n, sum
?CONTINUE WHILE

DUP S->F 1/F (--fn::--sum1/n)
SWAP FDUP FSIGN \transfer sign

 

 

F+ \sum = sum +

term

NEGATE \ flip sign of f
REPEAT

 

 

 

Fig. e2 Computing 3/4 by the infinite series for tan" (1)

 

5. Since iii/4 radians is 45°.
6. Ahettermethodwouldbetoevaluatetheseriesforx-l/flwhichisthetangentofu/Gmr
x-t/Z—l, the tangent ofn/S.)

Ware—Wm 121
The corresponding program is
15 #PIACES I \eetEtoiSdigits
:NEX'tTERM 12n+1--i'2n+3::sum--eum')
DUP S-> 1/F SWAP DUP FSIGN F+
NEGATE SWAP 2+ ;

:SEtUP FINIT F20 11
:EXHIBIT CR DUP
."n = ' . ZSPACES
FDUP F2‘ F2' .'4'sum .. ' F. ;
.'Anotherterm?' ?YN ;

: ?CONTINUE

: PI/4 SETIUP
BEGIN EXHIBIT
?CONTINUE

WHILE NEXITEFIM REPEAT ;

Now we run the program]:

FLOAD PIBY4 Loading PIBY4 0k

Pl/4

n -1 s‘eum - emmmw
n-3 sun-smmum‘rv
n - 5 4...!“ - zwmmrv
n - 7 ”sum - smmmfi
n - 9 Pain - zmmwn
n -114§I.m - ammunnrv
n - 13451111 - mnemonmmw
n -154'aun - mmmmrv
n . 17 451:» - 3.011071017117181 Moe-rm‘rY
n - 104m - WTIQ‘IMW‘IY
n - 21 o'er-n - 3m1m1mmm7¥
n-as‘aun-WIWWWTV

n - 254m - ammmvv
n - 274M - urmmmm‘rv
n - 34m - aomastmrsmm‘rv
n - 314m - mimmmww
n - 334'eum - sonrsmrenzmmrv
n - Sd'aum - emaimmanmv
n - 374m - smnmmmrv
n - sum - 3131m194mmm7Y
n - 414‘aum - swimmmiv
n - ess'eum - 3191mmm71
n - ‘54‘eum - seesiersasseaesmwmmrv
n - 4745mm - 3.1m153253mbrm7N air

The first thing we notice is that the numbers seem to be converg-
ing to something; however, unlike the previous series, the differ-
ences between successive partial sums are fairly large.

An infinite series of terms that alterna e in sign and decrease in
magnitude is. guaranteed to converge . The error (that is, the
difference between a partial sum and the limit) is of the order of

 

Note we have set up todisplayuratherthanxlt
'l'h'n theuent, due to Weierstrass, is found in all standard intermediate-level calculus tests.

OMMNoflelm-Mmm.

122 Chapter 8 - Programming Examples Scientific roam

the first neglected term and has the same sign. We see that for
this case, the error in computing rt is of order 2Jn, where n is the
number of terms in the sum. To get It to 3 significant figures,‘
therefore, we need about 1000 terms! This is why the series
representation for 4tan 1( 1)‘ rs not a very good way to calculate a.
A better way uses e. g. .,

= -11 _ -1;
7r 16tan (5) 4tan (239),

which converges much faster9.

§§3 The Infinite sum program
Evaluating a function from its infinite series representation
Eq. 1 provides an illustration both of indefinite loops and oi
tests of floating point numbers. We anticipate a program struc- .
ture something like this: "V

 

BEGIN
Calculate next term
Not converged?
WHILE
Update:
Add next term to sum
increment n
REPEAT

To actually write the program, we begin at the end, by specifyin, 3
how we want to invoke the function.

Functions in the standard FORTH library (Ch. 3 §3) typical]
expect a single real number on the fstack replacing it with th I
function: x -» {(1). Since the sum is infinite, we supply the coeif; 3,
cients c. as a function of n rather than as an array. For any give,
=f(r1), it is easy enough to write a FORTH word that evaluatd
it and leaves the result on the appropriate stack (87stack, fstacii
i
9. See, e.g., .I. Mathews and 11.1.. Walker, Mathematical Methods of Physics, 2nd ed. (WA. Ben-
jamin, Inc, New Jersey, 1970).

Grams-WE” 123

ifstack). For example, to evaluate the exponential via the power
series

I .2 .3
€x=1+—l+'2—‘+'3—!+... (9)

we would define the word NEXITERM as

:NEX'IZTEFIM (87: termxn--term'x/[n+1]x n+1)
F=1 F+ n-> n+1
FROT FOVE F/ (87:--xn+1term/[n+1])
FROT FUNDER F' F-ROT ;

“3-1 Function calls In FORTRAN
However, FORTRAN (as well as languages that emulate it)
achieves readability by passing arguments to functions in a list. In
fact, function names can also be passed in the argument list. 111115,
e.g.. a general FORTRAN program to evaluate a function by
summing an infinite series could be written

REAL FUNCTION XINFSUM(X. E. C)
C
C EVALUATE INFINITE POWER SERIES
C

EXIERNALC
REALX.E,C
SUM=0
TERM-1
N=1

1 SUM==SUM+TERM
TERM -TERM'X'C(N)
N =N +1
IFCFERM.LE.E)RETURN
GOTOt
EM)

124

Chapter 6 — Proonmmhg Examples Scientific FORTH

where the program to evaluate the exponential would be

REAL FUNCTION EXPOt)
EXTERNAL )(INFSUM. OOEFEXP
REAL co. XINFSUM

COMMON ICBLK/OO

co = 1

EXP = XINFSUM(X,1.E-7,00EFEXP)
RETURN

END

given the coefficient function

REAL FUNCTION OOEFEXP(N)
COMMON ICBLK/CO

co = OO/N

OOEFEXP = 00

RETURN

END

§§3-2 A function erI tor FORTH

We would like to extend to FORTH FORTRAN’s ability to write
a generic series summation function, passing the variable and the
name of the coefficient function as arguments. In other words, we
now face the task of devising the function protocol we plan to use
throughout the rest of the book and in our future programming
— a heavy responsibility since we do not know what form these
future programs will take. For once we must engage in top-dawn
programming!

We want our protocol to have several features:
s it must be telegraphic, i.e. it must immediately suggest what it

is doing — a matter of choosing good names.

e it must be simple to implement and easy to remember — the
advantages of using it must not be outweighed by complexity.

e it must be fast — a major drawback to FORTRAN's way ol
doing things is the overhead in function calls.

e It must be portable — it cannot depend on specific details 0
the FORTH implementation or the machine 1t 1s running on.

It is simpler to thread this particular maze in reverse: begin with
where we want to end up, and determine what steps got us there.
We suppose we have defined a generic power series summation
function, using the lfstack defined in Ch. 5 52595:

: SUMPOWERS (87zerr-- 21x --surn)
E G! \storeerror
FS>F DUP F>FS \gettypeofx
DUP G=1 \x“0
G=0 0 DUP --n=0 ::--x 10)
adr.c EXECUTE G+ --0::--x1c[0])
BEGIN 1+ --n::~-xx“n—13um)
FS>F GOVER G'
(--n87:--sum::--xx“n)
DUPadr.c EXECUTE
(--n::--xx“nc[n])
3 GPICK G' F>FS
(--n::--xx“ntermsum)
ENUF? NOT WHILE
G+
REPEAT G+ CLEANUP ;

AAA

The words E, ENUF? and CLEANUP have straightforward
definitions with obvious meanings; mine has not been defined
because we have yet to figure out what it will do.

Manifestly, edr.c must place the execution address (“code-field
address" — cfa) on the stack for EXECUTE to find. There are
several ways to accomplish this. Clearly, we want to keep the
variable that holds the cfa local, so it will not get confused with
another function’s cfa. While it is straightforward to define a
variable and then make it headerless —hence local— (via
BEHEAD", e.g.) we would prefer to avoid defining a variable at
all. Here is a perfect opportunity to use the return stack (rstack).

We imagine that SUMPOWERS expects the cfa of the function
c. on the stack. Then the first thing SUM.POWERS must do is
stash the cfa somewhere convenient but local. One such place is
the stack itself. Even easier, since SUMPOWERS does not use
the rstack explicitly (e.g. in a DO LOOP), is to let the first step
in SUMPOWERS be > R. Then the code for adr.c would be
merely R@. The final word, after invoking CLEAN.UP (that
drops unwanted items from the various stacks), then must be
RDROP (for systems without it, : RDROP R > DROP ; ).

owvmim-Mmm.

126 Chapter 6 - Programing Enmdoc Scientific FORTH
The revised version of SUM.POWERS is then
: SUM.POWERS (adr.c - - 87: err - - :: x - - sum)
> R \adr.c -> rstack
E GI \ store error
FS>F DUP F>FS \getx's type
DUP G=1 \x“0
6:0 0 DUP (--n= 0 ::--x10)
R@ EXECUTE G+ (- -0 ::--x1 c[0])
BEGIN 1+ (--n::--xx“n-1sum)
(

FS>F GOVER G'
DUP R@ EXECUTE (--n::
3 GPICK G' F>FS(--n::

GOVER
ENUF? NOT

--n87: --sum:
--xx“n O[n])

WHILE G+ REPEAT

- -xx“r

- - x x**n term sum)

G+ CLEANUP RDROP;

The full program to evaluate the exponential by summing power
series would then have the form shown below:

 

\ evaluate exponential by summing series

DROP FS >F
GDROP GDROP F>FS :

ELSE c G@ S->F
F‘ FDUP c G! 1/F

 

 

THEN REAL’B F>FS;

REALM SCALAR E BEHEAD' c \hidethisvariahie

: ENUF? (::term--) (--1) : USE( [OOMPILE] ' CFA LITERAL;
GABS FS>F EG@ F>; IMMEDIATE

:CLEAN.UP (n-- ::xx“nsum--sum) \thlsmeam'tseaddressot'

\--cruc'aldef'noifunctionle)dcon
:E“X (::x--e"x) %1.E—B

\definition of SUM.POWERS iromabove \erroron87stack
BEHEAD" E CLEANUP \hidethese def'ns USE( COEFEEXP
REAL‘B SCALARC F=1 CG! \adr.constack
: COEEEXP (n--) (z: --c[n]) SUMPOWERS ;

DUP 1 < =

IF F=1 c G! F=1 DROP

 

‘1 SMFORTH

MO-Wm 127

52 “anecenderlalequations

A transcendental equation has the form

flx) = 0, (10)

where f(x) is a transcendental function rather than, say, a polyno-
mial or ratio of polynomials10 (rational function).

There are several standard methods for finding a (or possibly, the)
value of x that satisfies this equation, i.e. a root of Eq. 10 (which
might have many or no roots). '1?) guarantee that we find a root
we must know an interval of the x-axis that it certainly will be
found in. 'Iirr'o methods that can be applied under these circum-
stances are binomial search and regula falsl .

ssr Binary search

Let us look first at binomial search, since its algorithm is euy to

understand. We know some interval, It. s x 5 xx, contains a
root because f(x) changes sign when .1: goes from xL -xn . In
pseudocode (and FORTH flow chart) the binomial search algo-
rithm is shown in Fig. 6-3 on page 128.

The method begins with upper and lower bounds on x that capture
the root. Next we look at f(xAV) halfway between xL and xglf
f =f(xAV) has the same sign as fL= f(xL), the new left end of the
interval becomes X". If the signs are opposite, xAV becomes the
new right end of the interval. The algorithm is done when left and
right ends agree within some predetermined accuracy.

Binary search has the following virtues: the time it takes to
achieve a given accuraqr is predictable, and it is guaranteed to
find a captured root. Creating a FORTH program from the
pseudocode skeleton of Fig. 6-3 is left as an exercise.

 

10. Wespedafiumuamcendentdequfionsheauseomrou-findingmethodswiflwkakofa
pdynomhlgwherusthemethodsdevelopedfmpdynomhkfiflnmwakhthemegeurfl

GJUIanVNohieiflz-Almreeuved.

Were-PWW SclmtlflcFORTH

 

INITIALIZE
ft = f(XL) fa =11an

$9 BEGIN

|xL — xn| > error ?
WHILE

x=%(xL+xR), f=!(x)

 

sign(f) =sign(fL) ?

 

 

IF XL =X fL =f

 

ELSE xn =x in =f

 

 

 

 

 

 

THEN

 

REPEAT

 

 

Flg. 6—3 Binary search algorinn for roots of f(x)

§§2 Regula falsl
Now we look at regula falsi, Latin for “rule of false approach"
Here the basic premise is:

0 Assume the root lies in the interval (x1, x3, and plot a straigli
line between the points (XL, fL) and (XE, f . l

o This line must intersect the_x-axis somewhere in the inte
and we take that point, call It x’, as our next guess.

a If x’ is to the left of the root, ad'ust the interval accordingly, an
the same if x’ is to the right 0 the root.

As Fig. 6-4 on page 129 shows, the straight line is supposed
approximate the curve [(15). The new guess may be much closer

the root than is the midpoint of the interval (which was the neat
guess in binomial search).

 

 

 

 

 

 

 

Fig. 6—4 Graphical Illustration of ragula falsl

A straight line in the x-y plane has the analytic form
y = at + b (11)

where a and b are constants. The intercept of the straight line with
the x-axis is gotten by setting y = 0 and solving for x:

,=_-2

x a (12)
To determine a and b we use the two equations

fl. = 01L + b

(13)

In = Uta + b

giving
1 _ ma
0 2|:fL+f|l ‘l_xL(-1'L+Ia)] (14)

emvuouum—me.

130 Word—Programme. SclentlflcFORTH

and thus

x.=m
fa‘fi. '

(15)

A FORTH flow diagram for this algorithm appears in Fig. 6-5.

 

<8

 

 

 

 

 

 

 

 

 

 

 

 

 

 

INITIALIZE
ft=f(XL) fn=f(xi=i)
Xoid = Xi.
BEGIN calculate x’
|x' — x°.d| > error
?
waits
f= '(X’)
sgn(r") = sgn(fL) ?
IF XL =X fL =f

ELSE Xn =X TR =f

THEN

 

REPEAT

Fig. 6-5 Regula falsl algorithm for roots 0! f(x)

The corresponding program is shown on page 132. 3

Here is an example of the program in action:

FINIT 6 #PLACESI ok \set display to 6 digits
\ Example: f(x) = exp(-x) - x

: FNA FDUP FNEGATE FEXP FR— ;

\ Cleariy, the root lies between 0 and 3. (Why?)

macaw me-Wsm 131

USE(FNA 950. 963. 961.56 )FALSI
\st(4) “(3) “(2) x' x' - xoid
77???? 777??? 777??? .759452 ..291530
777??? 77???? 7'27??? .588025 -.0326025
777??? 777??? 7?)??? .569459 -.00362847
77???? 77???? 77???? .567400 -.000403560
77???? 77???? 77???? .567171 -.oooo443740
?77??? 77???? 77???? .567146 -.0000050002
72???? mm 77???? .567143 -.0000005544 ok

The display was generated by the word .FS —placed in the
definition of APART? for debugging— we show the top 5 of the
eight 80x87 registers. The ?????? means the contents of that
register are not a properly defined fp number, either because of
a mistake or because nothing was stored in them after FINIT.
Here, since the program is obviously working, the latter explana-
tion is the correct one.

53 Ordinary differential equations
We wish to solve the first-order general differential equation

:2 a g =flxr) (16)

In general we can only solve Eq. 16 approximately, starting from
the value of x - call itxo — at some initial time to , then advancing
the time by small increments d: =h, using the differential equa-
tion itself to give us x(t +h) given x(t).

For example, we could emand in Taylor’s series11

2
x(t+h) = x(t) + hi(t) + 52-35(1) + (17)

 

11. See, e.g., um [3.6.1.

132 Chapters-Program Enmplea

\USE( Fname 96a 96b 98m )FALSI
(87:--rod)
\meeisthemmeoiaFORTl-ihnctlon

\iunctionnotation
: USE( [COMPILEl' CFA LITERAL;
IMMEDIATE

6REAL‘4 SCALARS ERR XLXRYLYR OLDX

o VAR f1 \apiacetostorecia
:SAME.SIGN? (87:xy-- --i)
F' F0> ;

:INITIAIJZE (cia--87:abe--)
is it \storecia
XDUP \interval hasroot?
SAME.SIGN?
ABORT" Even number oi rootsil !"
XR G! XL Gl
XL G@ it EXECUTE YL G!
XR G@ 11 EXECUTE YRGI
F=0 OLDX G! ;

 

:X' XLG@ FRG@ (872"!)

FUNDER F' (crummy!)

XR G@ YL G@

FUNDER ma :--yaxvynyi_xn'ytiil
FROT F- (a7:--ynyi_m-yt-n.'yn:

F-ROT F- Fl ;
:APART'? (arr-qr --o

FDUP OLDX G@ F-

.FS FABS ERR G@ F>;

: REVISE (87:x'--)

1
FDUP f1 EXECUTE (872--X'Y)

FDUP YL G@

SAMESIGN? FOVER

(--f 87:--x'y' x')

IF XL 6! YL G!

ELSE XR G! YR G! THEN
OLDX G! ; (-- 87: --)

Sclenflflc FORTH "

d
I

 

:)FALS| (cfa-- 87:abe--noa) INITIAIJZE ;

and keep only the lowest order terms:

x(t+h) z x(t) + hf(x(t), t) .

§§1 Runge-Kutta method

One stande class of methods that had fallen into disfavor bi,
now are popular again, are the Runge-Kutta algorithms”. Tit '
algorithms can be classified according to order n (that is, if h ,

(13)

i

 

l

the step size, the error at each step will be 00:"). The seco Er
order Runge-Kutta algorithm is (x' E x(t+h) , x E x(t) ) ‘

 

12. "MP, £25.16.

Mil-WE” 133

k ' him) ( 9)
1

x' ax + %(k + hflx+k, (+h)) + 00?).

How does this work? Clearly,

k + hflx+k,r+h) .- Mix!) + Mb, I) + ,3?! + hk %

(20)
a 2m) + #360) + 00?)

Substituting 19 in 20 we now find

2
x’ = x(t+h) = x(t) + ruin) + ”3550) ;

that is, the Rur¥e-Kutta x' agrees with the Taylor's series expan-
sion 17, to 00: ).

The flow chart of second-order Runge-Kutta is shown in Fig. 6.6
below. We express the algorithm in FORTH as shown in Fig. 6-7
on page 134 below.

 

 

)RUNGE
INITIALIZE: get h, ii, to , x0

BEGIN DISPLAY
DONE? NOT
WHILE
STEP
REPEAT

 

 

Fig. 6-6 2nd-order Runae-Kutta for dxldr = f(x,t)

134

Chapter 8 - Programming Examples

Scientific Foam;

 

\ STRAIGHT 2ND ORDER RUNGEmrTA
\ SOLUTION or: FIRSTORDER DIFEQ
\ dx/dt = i(x,t)

\ See Abramowitz 8i Stegun. HMF 525.5.6

\Usege:

\ USE( FNB 96x0 960 96d 96h )RUNGE

\

\ FNB (: [t]-- 87:x--t[x,t]) evalmtesi(x.t)

\ x0 = starting valueoidep. var‘abIe

\ t0 =initiaItime

\ ti = endtime

\ h = step.size

\

\ k = hi(x,t), x' = x + (k +hl(x+k,t+h))/2

6REAL‘4 SCALARS T T' H X TMAX K

0 VAR i1 \tohoid cia

: USE( [COMPILE]' CFA LITERAL;

IMMEDIATE

: INFI'IAUZE (:cta--87:)010tth--)

ler HG! TMAXG! TG! XGl;

 

 

\Theeeworda Iricremerxxait.
:lnc.T T G@ HG@ F+ T'GI;

:Inc.X X G@

T i1 EXECUTE (a7:--f[x,t])

H G@ F' (87:--k=hi[x,t])

FDUP K G! \savek

XG@ F+ (87:-- x+k)

T' G@ T G!

T 11 EXECUTE (87:--i[x+k,t+h])

H G@ F'

KG@ F+ F2/
(87:--[k+i[x+k,t+h] 1/2)

X G@ F+ X G!

:DONE? TG@ TMAXG@ F); (:--f)

 

o VAR exact \cia

: DISPLAY exact EXECUTE
XG@ T G@ CR F. F. F.

\emit "t x exact”

:)RUNGE (:cfa-- 872x010Uhn)
BEGIN DISPLAY
DONE? NOT

 

Fig. 6—7 Explicit 2nd-order Runge-Kutta solver

As an example of second-order Runge-Kutta in action, let us 50le
numerically the equation ,

5c=te

(21) l

with the initial condition x(t = 0) = 0 , whose exact solution is

x(t) =log,(1 +%13).

(22) 1

t x h. t x 1..
.0000 .0000 .0000 2.51:» 1 .0254 1 .0250
.10000 01150 .oooaa am am 2.02:!)
.20000 .oosoo ones 27:» 2.1101 211.
30000 0“” come am 22113 22115
.40000 .02177 02111 am 23023 2.325
.50000 .04104 .04002 3.09:; 23912 2.3015
.00000 .07040 cm 3.1m 2.4701 2.4704
.70000 .10034 .1002!) am 2501‘!) 25033
.30000 .15075 .15757 asses 2.0459 2.0462
90000 .21077 .21752 3.4M 2.7270 2.7273
1 .011!) 20% .20760 3.5% 2&1 2.305
1 .111!) .30046 .30710 am acme 20KB
1 2000 .45012 454$ 37% 2%92 2.9593
1 .3000 .ssoaa .54046 am 3.0333 30%
1.41110 .0331 6464 3.9999 3.1057 3.1M
1.5000 .75473 .75377 4.0% 3.1700 3.17m
1.01110 .w175 .“1 4.199 324m 32403
1.71130 97(51 seam 42999 3.313 3.3142
1.9000 1.0810 1.0797 4.3% 3.3m4 3.3330
13000 1.1312 1.197 4.4999 3.4450 3.4460

2.0000 1 2% 1 .2992 4.5999 3.5% 3.5099
2.1M 1.40110 1.4070 4.6% 3.5722 35725
22000 1.5151 1.5149 4.79% 3.63:5 3.63:3
2.299 1.6% 1.626 4.0% 3.6%9 3.0942
2.3999 1.7242 1.7241 4.- 3.7531 37534
2.49% 1.0250 1.0250 5.0. 3.0111 30114

 

Fig. “Second-order Runge-Kutta — results
Thus define
:FNB (:[T]-- 87: x--f[x,t])
FNEGATE FEXP G@ F“2 F*

: EXACT r G@ FDUP FDUP F' F3 (87:--t"3)
3S->F F/ F=1 F+ FLN ;

and say

USE( EXACT IS exact ok
USE( FNB 16 o. 96 o. as 5. 16 0.1 )RUNGE

The resulting output is shown in Fig. 6-8 above.

emumrm-Mmm.

136 Ctnptere—ngramnthxanfiaa SclentlflcFORTH

“2 An Implicit Runoe-Kutta formula
A variationl3 on straight Runge-Kutta is a so-called Implicit
algorithml3 .For example, in the second-order formulae given
above, supposex + k were replaced byx’:

’6 = Min) (23)

x’ = x + a); + hflx’, (+h)) + 00.3)

and the resulting (transcendental) equation solved forx’ by —say- -
regulafalsi. Since we have already written a regula falsr' program, .
we can apply" it here to get the algorithm shown diagrammatically
in Fig. 6- 9 below. We program 1 as shown in Fig. 6-10 on page i
137 below.

w...

 

 

)RUNGE
INITIALIZE: get h, if, to , x0

BEGIN DISPLAY

DONE? NOT
WHILE

k=hf(x,t)

SOLVE: x' = x + k/2 +hf(x’,t+h)/2
REPEAT

 

 

 

i’
Fig. 6-9 2nd-order impiicit Runge—Kutta for (tr/d! = f(x,t) i
l
l
i
i
l
l
,

 

13. See, c.g. . A. Ralston,A First Course in Numerical Analysis (McGraw-Hill Book Company, New
York, 1965) Ch. 5. Implicit methods increase the stability of numerical solution, compared with
explicit methods. The formula below is exact for second-order polynomials. The error is of the
same order as the explicit formula, but the eoeffiient may be smaller.

14. m “ ?( " means “conditionally execute to next right parenthesis”.

WM?" Ware—PW!” 137

Now we consider the same example as previously:

:FNB (:[Ti-- 07:x--f[x,t])
FNEGATE FEXP G@ F"2 F' ;

: EXACT T G@ FDUP FDUP F' F' (07: --t"3)
F-3 F/ F=1 F+ FLN ;

and say

USE( EXACT IS exact ok
USE( FNB as o. 91. o. as 5. 91. 0.1 )RUNGE ok

 

FIN) )FALSI 0:- ?(FIDADFALSIFTH) XG@ F+
7REAL'4 SCAIARS TT' H XX'TMAXK 963. FOONSTANT F—a
OVARn \tohoidciaotiou) :INTERVAL XG@ FDUP
:IMTIAIJZE ISfl KG@FsaF' F+;
HGI TMAXGI T61 X61;
:X' USE( X" lNTERVAL 961.56
\‘iheaewords lneremetlxait )FALSI ;
:lnc.T TG@ HG@ F+ T'Gl; :lnc.X kX’ XGIT'G@TGI;
:kXG@ :DONE? TG@ TMAXG@ F>;
T 11 EXECUTE(87:--i[)tt]) (:--o
H G@ F‘ KG! ; (57:--k=iir{x,q) OVAR exact \cia
:DISPLAY exactEXECUTE
:X" (07:x'--g[x’]) XG@ TG@ CR F. F. F. ;
1" i1 EXECUTE (07:--i[x'.t+h]) \eml'tx exact'
HG@ F“ KG@ F+ F2] :)RUNGE (:da--87:XontIh--)
(07:--[k+fix+k,t+h] 1/2) BEGIN DISPLAY

 

 

 

 

Fig. 6-10 lrnpllclt 2nd-order Runge-Kutta program

The resulting output is shown in Fig. 6-11 on page 138.

Clearly the implicit form is more accurate; whether the putative gain
in stability justifies solving a transcendental equation is unclear,
however.

138

 

ChapterB-ProgrammlngExampiea Scientific-FORM:
t X 11.,l t x X.
00000 .00000 .00000 2.5% 1.9253 1.9255
.10000 .(XXJ50 0003:! 2mm 2.0228 2.0230
.20000 .00299 .m266 2.7909 2.1181 2.1183
.30000 .(XJ945 DOM am 2.2113 2.2115
.40000 .02173 .02111 am 2.3324 2.3.125
.50000 .04155 .04082 3.0% 2.3914 2.315
.60000 .07032 1353 3.1% 2.4783 2.4784
.TWO Am .10825 3.2% 2.5632 2.5633
.BCXJOO .15834 .15757 3.39% 2.6461 2.6462
.90000 .21822 .21752 3.4% 2.7272 2.7273
1.0000 .28825 .28768 3.59% 23164 2.“
1.1000 .36762 .36718 3&9 2.8838 2."
1.2000 .45518 .45489 3.7% 2.9595 2.596
1.3000 .54962 .54946 3.0999 3.0336 30005 i
1.4000 64%8 .64954 3.99% 3.1060 3.1m ‘
1.5000 .75370 .75377 4.“ 3.1769 3.1769
1.0000 .%O77 06191 4.19% 3.2463 3.2463
1.7000 .m .WBS 4.2% 3.3142 3.3142
1.&)00 1.0795 1.0797 4.3% 3.3838 3.3000
1.9000 1.1%5 1.1897 4.49% 3.4460 3.4460
2.0000 1.2%0 1.2992 4.5999 3.5099 3.5099
2.1000 1.4075 1.4070 4.3:.- 3.5725 3.5725
2.2000 1.5147 1.5149 4.7% 35340 3.6339
2.29% 1.6202 1.6205 4.0999 3&42 3&42
2.3999 1.7239 1.7241 4.9Q9 3.7534 3.7534
2.4% 1.8256 1.8258 5.0% 3.8114 3.8114

 

Fig. 6-11 Second order Wunge—Kutta — results

Let us now compare the two algorithms for functions f(x,t) th:
lead to singular solutions. This time we consider the equation

1:1. (20'

whose exact solution is

x(t) = -log,(1—%t3). (25)

Manifestly, 25 blows up at 1= 3"“ =1.442... We expect 111
behavior to become apparent in the numerical solution. So we s

 

OJUIanV. Noble1992-Alirightsreserved.

awe-WW 139

:FNB FEXP G@ F"2 F';
([t] -- 07: x --i[x,t])

:EXACT F-1 TG@ (07:--t"3)
FDUP FDUP F’ F' F-3 Fl F+ FLN -

1

and say again

USE( EXACT iS exact ok
USE( FNB 96 o. 91. 0. 91. 5. 91. 0.1 )RUNGE ok

The results of doing this with straight-, and then implicit Runge-
Kutta are displayed in Fig. 6-12 (p. 140) and 6-13 (p. 141), respec-
tively. We only show the second half of the interval (near the
singular point) in either case.

The straight Runge-Kutta algorithm, without the fancy implicit
solution for r(t +h), appears more accurate near the singularity,
although both methods are acceptably accurate. Does this mean
implicit Runge-Kutta is no good? No!

The implicit scheme lost accuracy through roundoff: the arith-
metic was insufficiently precise. To take advantage of the
method’s power, we must increase the precision beyond one part
in 106. This requires changing all scalars to 64—bit precision
(REAL‘S) rather than 32-bit as we have done here. The generic
fetch/store techniques developed in Chapter 5 and used here,
permit this change with a minimum of fuss. We leave this as an
exercise.

OJtlthNobietm—Alridaamerved.

140

Mud-PWW

a-o

1.01%
1.02%
1.03%
1.04%
1.05%
1.0%
1.07%

X
.13287
138%
.14512
.15156
.15&1
.16509
.17220
17%5
.18714
.194%

.21 147
.22012

238%
247%
257%
.26788
.27841
28928

.31215
.32417

.48757
505%
.5245
.54458

Fig 6—1281ralghtRunge-Kutta forslngularease

132%
.13888
.14511
.15154
.15Q0
.16508
.17219
.17%4
.18713
.19497

21145
22010

23828
.24782
25767
267%
.27 K9

.30051
.31213
.32415
.33657
.34943
.36272
.37648
.3%72
.40546
.42073
.4354
.4523!
Am
.48755
50584
524%
.54456

 

1.0m
1.0m
1.10m
1.11%
1.129
1.139
1.14%
1.15%
1.1m
1.17%
1.109
1.1%
1.20%
1.21%
1.22%
1.23%
124%
1.25%
1.2%
1.27%
121m
1.29%
1.3%
1.31%
1.32%
1.33%
1.34%
1.35%
1.3M
1.37%
1.3a
1.3%
1.4M
1.41%
1.42%
1.4m

OJUIMVNobIetfl-Almw.

x ”at
W 50500
50040 same
mean .0057
..171 001$
new .$578
.0003: .6331
.70718 .70715
.73461 .73458
.m1 .m
.79007 .79335
.&491 .&4%
05000 .5531
.%287 .W
92959 ms
.W 95004

1.00% 1 mm
1 .0527 11527
1 09% 1 as.
1 .1481 1 .14&
1 .2307 1 21118
1 2571 1 .2572
1 .31 7'9 1.31 m
1 .3815 1.3837
1 .4550 1 .4552
1.5332 1 .5334
1.61 % 1 .61 %
1.7151 1 .7154
1 .&26 1 .8231
1 .9449 1 .9457
2.0%5 2.1376
2.5% 2.557
2.4581 2.461 1
2.71 87 2.7242
3.07m 3%
3.76 3.6782
4.8831 5.1557

Mao-WW

1 x 11.: 1 x for

.71“ .132U .132! 1.0m .54464 .54466
.739 .13“ .13” 1.0a .515 .561!
.7“ .14613 .14511 1.01m m .m
.74“ .15156 .15154 1.101» .m 30057
.75“ .1592 .15820 1.11% .631!) .6319
.7“ .1510 .1506 1.129 55590 .66578
.77“ .17221 .17210 1.1399 .66104 .6831
.7“ .1756 .1754 1.14m .70729 .70715
.7“ .16715 .18713 1.15m .73473 .73458
.aanoo .19500 .19407 1.1m .7“ 7&129
mm 20310 2m 1.1m 78353 19995
.62“ 21146 .21145 1.1399 82.506 .8243
33999 .22013 .22010 1.1m 35822 .858”
114999 22!)? 22904 1.2099 .asaoo 39285
.85” 23631 23828 1.21% .m 92%6
seas .2475 24782 1229 .96w1 36804
37999 25771 25767 1m 1.00% 1.0090
.awaa .267. 267% 1.2499 1.0531 1.0527
.me 27842 27839 1.25% 1.0993 1.0939
.5“ 2K!) 2&26 1.2699 1.14% 1.1482
91% 30055 .30051 1.2799 1.2013 1.2008
.... .31217 .31213 1.2199 12578 12572
.9399 .32419 .32415 1.2% 1.31% 1.316)
949% .33m2 .3365? 1.3099 1.3645 1.3637
35999 .3494? .34943 1.31% 1.4561 1.4552
.9609 .36277 .36272 1.32% 1.5345 1.534
.97999 .37653 .37646 1.33% 1.6209 1.61%
sagas .39077 33172 1.34% 1.7171 1.7154
.9999 .40551 .40546 1.3599 1.8252 1.8231
1.00m .42078 42073 1.3m 1.94% 1.9457
1.0199 Am .43654 1.3799 2.0914 20676
1.- 45300 .4583 1.3899 2.2612 2.2557
1.03% .4” Am 1m 2.4% 24611
1.049 .46762 .46755 1.4099 2.7395 2.7242
1.050 51592 50564 1.4199 3.1222 3.0664
1.0m .52491 52463 1.42% 3.8149 3.6782

 

Fig. 6—1 3 Implicit Range-Kym for singular case

142

Chane-WW

OJUMVNounm-Almm.

