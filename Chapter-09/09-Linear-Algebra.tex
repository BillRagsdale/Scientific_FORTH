% 9 — Linear Algebra
% :ontonts

% \S1 Simultaneous linear equations
% 551 Theory of linear algebraic equations
% \S\S2 Eigenvalue problems
% 52 Solving linear equations
% \S\S1 Cramer's rule
% \S\S2 Pivotal Elimination
% 553 Testing
% \S\S4 Implementing pivotal elimination
% \S\S5 The program
% 556 Timing
% 53 Matrix inversion
% \S\S1 Linear transformations
% 5552 Matrix multiplication
% \S\S3 Matrix inversion
% \S\S4 Why invert matrices, anyway?
% \S\S5 An example
% 
% 54 LU decomposition 214 214 215 218 218 221 225 227 227 242 242 243 244 245 245 246 213 214
% 
% 51 Simultaneous linear equations
% 
% \S\S1 Theory oi linear algebraic equations
% 
% ChapterB-LhearNgebra Sewncronr'

      
 
   
 
\chapter{Linear Algebra}

Two common problems in scientific programming are the
numerical solution of simultaneous linear algebraic equations
and computing the inverse of a given square matrix. This is such
an important subject As an exercise in FORTH program
development we shall now write programs to solve linear equations and invert matrices.

\section{Simultaneous linear equations}
\TallC{ We } begin with a dose of mathematics (linear algebra), and 
then develop programs. Several actual programming session
are reproduced \textit{in toto} --mistakes and all-- to give the flavor of
program development and debugging 1n FORTH.
\subsection{Theory oi linear algebraic equations}
\TallC{ We } begin by stating the problem. Given the equations
\begin{equation}
    \sum_{n=0}^{N-1}A_{mn}x_n=b_m \;. \label{eq:09_01}
\end{equation}
--\,more compactly written $\overleftrightarrow{A} \cdot \vec{x} = \vec{b}$\,-- with known \textbf{coefficient
matrix} $A_{mn}$ and known \textbf{inhomogeneous term} $b_m$: under what
conditions can we find unique values of $x_n$ that simultaneously
satisfy the N equations 1?

The theory of simultaneous linear equations tells us that if not all
the $b_m$ 's are 0, the necessary and sufficient condition for solvability is that the \textbf{
    determinant}\sepfootnote{09_01} of the matrix $\overleftrightarrow{A}$ should not be 0.
Contrariwise, if $det(\overleftrightarrow{A} )= 0$, a solution with $\vec{x} \neq 0 $ \textit{can} be found
when $\vec{b}= 0$.

\subsection{Eigenvalue problems} 
\TallC{ Many } physical systems can be represented by systems of linear 
equations. Masses on springs, pendula, electrical circuits,
structures\sepfootnote{09_02}, and molecules are examples. Such systems often can
oscillate sinusoidally. If the amplitude of oscillation remains
bounded, such motions are called stable. Conversely, sometimes
the motions of physical systems are unbounded -- the amplitude
of any small disturbance will increase exponentially with time. An
example is a pencil balanced on its point. Exponentially growing
motions are called -- for obvious reasons -- \textbf{unstable}.

Clearly it can be vital to know whether a system is stable or
unstable. If stable, we want to know its possible frequencies of
free oscillation; whereas for unstable systems we want to know
how rapidly disturbances increase in magnitude. Both these
problems can be expressed as the question: do linear equations
of the form
% (2)
\begin{equation}
    \overleftrightarrow{A} \cdot \vec{x} = \lambda \overleftrightarrow{\rho} \cdot \vec{b} \label{eq:09_02}
\end{equation}
have solutions? Here $\lambda$ is generally a complex number, called the
\textbf{eigenvalue} (or \textbf{characteristic} value) of Eq. \ref{eq:09_02}, and $\overleftrightarrow{\rho}$ is often called
the \textbf{mass matrix}. Frequently $\overleftrightarrow{\rho}$ is the unit matrix \textbf{I},
% (3)
\begin{equation}
I_{mn}=
\begin{cases}
     1, & m = n\\
     0, & m \neq n,
    \end{cases}
    \label{eq:09_03}
\end{equation}
but in any case, $\overleftrightarrow{\rho}$ must be \textbf{positive-definite} (we define this
below). A non-trivial solution, $\vec{x} \neq 0$, of the equation

% (4)
\begin{equation}
    \label{eq:09_04}
    \overleftrightarrow{A} \cdot \vec{x} = 0
\end{equation}
exists if and only if $||\overleftrightarrow{A}||=0$ . This fact is useful in solving \textbf{eigenvalue} problems such as Eq. \ref{eq:09_02} above.

% 216 

The \textbf{secular equation} (or determinantal equation)

% (5)
\begin{equation}
    \label{eq:09_05}
    || \overleftrightarrow{A} - \lambda \overleftrightarrow{\rho} || = 0
\end{equation}
is a polynomial of degree N in $\lambda$, hence has N roots (either real or
complex)\sepfootnote{09_03}. When $\overleftrightarrow{\rho}=\overleftrightarrow{I}$, these roots are called the \textbf{eigenvalues}
(or "characteristic values") of the matrix $\overleftrightarrow{A}$.

Eigenvalue problems arising in physical contexts usually involve
a restricted class of matrices, called \textbf{real-symmetric} or Hermitian
(after the French mathematician Hermite) matrices, for which
$A_{mn}^{*}=A_{nm}$ . (The superscript * denotes complex conjugation -- 
see Ch. 7.) All the eigenvalues of Hermitian matrices are \textit{real}
numbers. How do we know? We simply consider Eq. \ref{eq:09_03} and its
\textbf{complex conjugate}:
% (6a)
\begin{subequations}
    \begin{align}
        \label{eq:09_06a}
        \sum_{n} A_{mn} x_n &= \lambda \sum_{n} \rho_{mn} x_n ;\\
        \label{eq:09_06b}
        \sum_{n} x_n^* A_{nm}^* &= \lambda^* \sum_{n} x_n^* \rho_{nm}^* ;
    \end{align}
% (6b')
Equation ~\ref{eq:09_06b} can be rewritten (using the fact that $\overleftrightarrow{A}$ and $\overleftrightarrow{\rho}$are Hermitian)
    \begin{equation}
        \label{eq:09_06c}
        \sum_{m} x_m^{*} A_{mn} = \lambda ^{*} \sum_{n} x_m^{*} \rho_{mn}^{*} 
    \end{equation}
\end{subequations}
 


Multiply \ref{eq:09_06c} by $x_m$ and \ref{eq:09_06a} by $x_m^*$, sum both over at and subtract
this gives

% (7)
\begin{equation}
    \label{eq:09_07}
    0 = \Big(\lambda^*-\lambda \Big) \sum_{n,m} x_m^{*} \rho_{mn} x_n\;. 
\end{equation}

However, as noted above, $\rho$ is \textbf{positive-definite}, i.e.
% (8) 
\begin{equation}
    \label{eq:09_08}
    x^\dagger \cdot \rho \cdot x \equiv
    \sum_{n,m} x_m^{*} \rho_{mn} x_n > 0
\end{equation}
for any non-zero vector\sepfootnote{09_04} $x$. Thus, Eq. \ref{eq:09_07} $\lambda^*\equiv\lambda$, that is, $\lambda$ is real.

In vibration problems, the eigenvalue $\lambda$ usually stands for the
square of the (angular) vibration frequency: $\lambda=\omega^2$. Thus, a positive eigenvalue $\lambda$ corresponds to a (double) real value, $\pm\omega$, of the
angular frequency. Real frequencies correspond to sinusoidal
vibration with time-dependence $\sin(\omega t)$ or $\cos(\omega t)$.

Conversely, a negative $\lambda$ corresponds to an imaginary frequency,
$\pm i \omega$ and hence to a solution that grows exponentially in time, as

% (9)
\begin{align}
    \label{eq:09_09}
\sin(i\, \omega t) = i \sinh(\omega t)\nonumber \\
\\
\cos(i\, \omega t) = i \cosh(\omega t)\nonumber
\end{align}
There are many techniques for finding eigenvalues of matrices.
If only the largest few are needed, the simplest method is iteration: make an
initial guess $x^{(0)}$ and let
\begin{equation}
    x^{(n)}=\frac{A x^{(n-1)}}{\Big(x^{(n-1)}, \rho x^{(n-1)}\Big)^{^1/_2}}\nonumber
\end{equation}
Assuming the largest eigenvalue is unique,the sequence of vectors $x^{(n)} , n = 1, 2,\dotsc$, is guaranteed to converge to the vector
corresponding to that eigenvalue, usually after just a few iterations.

If \textit{all} the eigenvalues are wanted, then the only choice is to solve
the secular equation \ref{eq:09_05} for all N roots.

\section{ Solving linear equations}
\TallC{ The } test-and-development cycle in FORTH is short compared
with most languages. It is usually easy to create a working
program that subsequently can be tuned for speed, again much
% 218
more rapidly than with other languages. For me this is the chief
benefit of the language.

\subsection{Cramer's rule}

\TallC{ Cramer's } rule is a constructive method for solving linear equations
by computing determinants. It is completely impractical
as a computer algorithm because it requires $O(N!)$ steps to solve
N linear equations, whereas pivotal elimination (that we look at
below) requires $O(N^3)$ steps, a much smaller number. Nevertheless Cramer's
rule is of theoretical interest because it is a closed form solution.

Consider a square $N\times N$ matrix A Pretend for the moment we
know how to compute the determinant of an $(N-1)\times(N-1)$
matrix. The determinant of A is defined to be
% according to  Josefg's comment this one is assumed to be Eq. (10) 
\begin{equation}
    \label{eq:09_10}
    \det(A) = \sum_{n=0}^{N-1}A_{mn} a_{nm} 
\end{equation}
where the $a_{mn}$'s are called \textbf{co-factors} of the matrix elements $A_{mn}$
and are in fact determinants of specially selected $(N-1)\times(N-1)$
sub-matrices of A.

The sub-matrices are chosen by striking out of $A$ the n'th column
and m'th row (leaving an $(N-1)\times(N-1)$ matrix). To illustrate,
consider the $3 \times 3$ matrix
% (11)
\begin{equation}
    \label{eq:09_11}
    A = 
    \begin{pmatrix}
        1 & 0 & 5\\ 
        3 & 2 & 4\\ 
        1 & 1 & 6
    \end{pmatrix}
\end{equation}
and produce the co-factor of $A_{12}$:
% (12)
\begin{equation}
	\texttt{TBD}\label{eq:09_12}
\end{equation}
We also attach the factor $(-1)^{m+n}$ to the determinant of the
submatrix when we compute $a_{nm}$.

\TallC{A determinant} changes sign when any two rows or any two 
columns are interchanged. Thus, a determinant with two identical rows or
columns is exactly zeros\sepfootnote{09_05}. What would happen to
Eq. \ref{eq:09_11} if instead of putting $A_{mn}$ in the sum we put $A_{kn}$ where
$k\neq m$? By inspection we realize that this is the same as evaluating
a determinant in which two rows are the same, hence we get zero.
Thus Eq. \ref{eq:09_11} can be rewritten more generally
% (13)
\begin{equation}
	\sum_{n=0}^{N-1}A_{kn}a_{nm}=||A||\;\delta_{km}\;.\label{eq:09_13}
\end{equation}

This feature of determinants lets us solve the linear equation
$A \dot x = b$ by construction: try
% (14)
\begin{equation}
	x_n=\frac{1}{\det(A)}\sum_{m}^{N-1}a_{nm}b_m.\label{eq:09_14}
\end{equation}
We see from Eq. \ref{eq:09_13} that Eq. \ref{eq:09_14} solves the equation. Equation \ref{eq:09_14}
also makes clear why the solution cannot be found if $\det(A) = 0$.

A determinant also vanishes when a row is a \textit{linear combination}
of any of the other rows. Suppose row 0 can be written

\begin{equation}
	a_{0k}\equiv	\sum_{m=0}^{N-1}\beta _m a_{mk}\;;\nonumber
\end{equation}
that is, the 0'th equation can be derived from the other N-1
equations, hence it contains no new information. We do not really
have N equations for N unknowns, but at most N-1 equations. The
N unknowns therefore cannot be completely specified, and the
determinant tells us this by vanishing.

As an example, we now use Cramer's rule to evaluate the determinant of Eq. \ref{eq:09_11}. We will write

\begin{align*}
	||A||&=A_{00}a_{00}+A_{01}a_{10}+	A_{02}a_{20}\\
	a_{00} &= 
	\begin{Vmatrix}
		2 & 4\\
		1 & 6
	\end{Vmatrix}\\
	a_{10} &= 
	\begin{Vmatrix}
		3 & 4\\
		1 & 6
	\end{Vmatrix} (-1)\\
	a_{20} &= 
	\begin{Vmatrix}
		3 & 2\\
		1 & 1
	\end{Vmatrix}\\
\end{align*}
The determinant of a $1 \times 1$ matrix is just the matrix element, hence
\begin{align*}
	a_{00} &=  2 \cdot 6 + (-1)\cdot 4\cdot 1 = 8\\
	a_{10} &=  -(18 - 4) = -14\\
	a_{20} &=  (3 -2) = 1\\
	||A||&= 1\cdot 8 + 0\cdot (-14)+ 5\cdot 1 =13\;.
\end{align*}
How many operations does it take to evaluate a determinant? We
see that a determinant of order N requires N determinants of
order N-l to be evaluated, as well as N multiplications and N-1
additions. If the addition time plus the multiplication time is $\tau$,
then
\begin{equation*}
	T_N\approx N(\tau +T_{N-1})\;.
\end{equation*}
It is easy to see\sepfootnote{09_06}\sepfootnote{09_07} the solution to this is
\begin{equation*}
	T_N =N!\tau \sum_{n=0}^{N}\frac{1}{n!} \underset{n\to\infty}{\rightarrow} N!\tau e\;.
\end{equation*}
In other words, the time required to solve N linear equations by
Cramer's rule increases so rapidly with N as to render the method
thoroughly impractical.

\subsection{Pivotal Elimination}
The algorithm we shall use for solving linear equations is elimination, just as we were taught in high school algebra. However we
modify it to take into account the experience of 40 years's solving
linear equations on digital computers (not to mention a previous
20 years's worth on mechanical calculators!), to minimize the
buildup of round-off error and consequent loss of precision \sepfootnote{09_08}.

The necessary additional step involves pivoting -- selecting the
largest element in a given column to normalize all the other
elements. This will be clearer with a concrete illustration rather
than further description: Consider the $3 \times 3$ system of equations:
%  (15)
\begin{equation}
	\begin{pmatrix}
	1 &0 &5\\ 
	3 &2 &4\\ 
	1 &1 &6
	\end{pmatrix}
\begin{pmatrix}
x_0\\
x_1\\
x_2
\end{pmatrix}
=
\begin{pmatrix}
    0\\
    4\\
    2
\end{pmatrix}
\end{equation}
We check that the determinant is $\neq0$; in fact $det(A) = 13$. The first
step in solving these equations is to transpose rows in \textbf{A} and in \textbf{b}
to bring the largest element in the first column to the $A_{00}$ position.

\underline{Notes:}

% (16)
\begin{itemize}
  \item The x's are not relabeled by this transposition. 
  \item We choose the row (n=1 --second row) with the largest (in
      absolute value) element $A_{n0}$ because we are eventually going
to divide by it, and want to minimize the accumulation of
roundoff error in the floating point arithmetic.  
\end{itemize}
Transposition gives
\begin{equation}
    \begin{pmatrix}
        3 &2 &4\\ 
        1 &0 &5\\ 
        1 &1 &6
    \end{pmatrix}
    \begin{pmatrix}
        x_0\\
        x_1\\
        x_2
    \end{pmatrix}
    =
    \begin{pmatrix}
        4\\
        0\\
        2
    \end{pmatrix}
\end{equation}
Now divide row 0 by the new $A_{00}$ (in this case, 3) to get
% (17)
\begin{equation}
    \begin{pmatrix}
        1 &^2/_3 &^4/_3\\ 
        1 &0 &5\\ 
        1 &1 &6
    \end{pmatrix}
    \begin{pmatrix}
        x_0\\
        x_1\\
        x_2
    \end{pmatrix}
    =
    \begin{pmatrix}
        ^4/_3\\
        0\\
        2
    \end{pmatrix}
\end{equation}

Subtract row 0 times $A_{n0}$ from rows with n $>$ 0
% (18)
\begin{equation}
    \begin{pmatrix}
        1 &^2/_3 &^4/_3\\ 
        0 &^{-2}/_3 &^{11}/_3\\ 
        0 &^1/_3 &^{14}/_3
    \end{pmatrix}
    \begin{pmatrix}
        x_0\\
        x_1\\
        x_2
    \end{pmatrix}
    =
    \begin{pmatrix}
        ^4/_3\\
        ^{-4}/_3\\
        ^2/_3
    \end{pmatrix}
\end{equation}

Since $|A_{11}| > |A_{21}|$ we do not bother to switch rows 1 and 2, but
divide row 1 by $A_{11} =-2/3$, getting
% (19)
\begin{equation}
    \begin{pmatrix}
        1 & ^2/_3 & ^4/_3\\ 
        0 & 1 & ^{-11}/_2\\ 
        0 & ^1/_3 & ^{14}/_3
    \end{pmatrix}
    \begin{pmatrix}
        x_0\\
        x_1\\
        x_2
    \end{pmatrix}
    =
    \begin{pmatrix}
        ^4/_3\\
        2\\
        ^2/_3
    \end{pmatrix}
\end{equation}
We now multiply row 1 by A" a 1/3 and subtract it from row 2,
and also divide through by A" to get
% (20)
\begin{equation}
    \begin{pmatrix}
        1 & ^2/_3 & ^4/_3\\ 
        0 & 1 & ^{-11}/_2\\ 
        0 & 0 & 1
    \end{pmatrix}
    \begin{pmatrix}
        x_0\\
        x_1\\
        x_2
    \end{pmatrix}
    =
    \begin{pmatrix}
        ^4/_3\\
        2\\
        0
    \end{pmatrix}
\end{equation}
The resulting transformed system of equations now has 0's to the
left and below the principal diagonal, and 1's along the diagonal.
Its solution is almost trivial, as can be seen by actually writing out
the equations:
% (21)
\begin{subequations}
    \begin{align}
         \label{eq:09_21_0}
        1 x_0 + 2/3 x_1 + 4/3 x_2 &= 4/3\\
         \label{eq:09_21_1}
        0 x_0 + 1 x_1 + (-11/2) x_2 &= 2 \\
        0 x_0 + 0 x_1 + 1 x_2 &= 0 \label{eq:09_21_2}
    \end{align} 
\end{subequations}
That is, from Eq. \ref{eq:09_21_2}, x; = 0. We can back-substitute this in  \ref{eq:09_21_1},
then solve for x, to get
\begin{equation*}
    x_1 = 2 - (-11/2)\cdot(0) = 2
\end{equation*} 
and similarly, from \ref{eq:09_21_0} we find
\begin{equation*}
x_0 = 4/3 - 2/3\,(2)+4/3\,(0) = 0 \quad .
\end{equation*} 
We test to, see whether this is the correct solution by direct trial:
\begin{align*}
1 \cdot 0+0 \cdot 2+5 \cdot 0=0\\
3 \cdot 0+2 \cdot 2+4 \cdot 0=4\\
1 \cdot 0+2 \cdot 2+6 \cdot 0=2
\end{align*} 

his works -- we have indeed found the solution.
% 224


How much time is needed to perform pivotal elimination? We
concentrate on the terms that dominate as $N\to\infty$.

The pivot has to be found once for each row; this takes N-k
comparisons for the k'th row. Thus we make $\approx$ N /2 comparisons
of 2 real numbers. (For complex matrices we compare the
squared moduli of 2 complex numbers, requiring two multiplica-
tions and an addition for each modulus.)

We have to divide the k'th (pivot) row by the pivot, at a point in
the calculation when the row contains N-k elements that have not
been reduced to 0. We have to do this for k = 0, 1, ..., N-1, 
requiring $\approx N^2/2$ divisions.

The back-substitution requires 0 steps for $x_{N-1}$, 1 multiplication
and 1 addition for $x_{N-2}$, 2 each for $x_{N-3}$, \textit{etc.} That is, it requires

and similarly, from \ref{eq:09_21_0} we find
\begin{equation*}
    \sum_{k=N-1}^{k=0}(N-1-k) \approx N^2/2
\end{equation*} 
multiplications and additions.

The really time-consuming step is multiplying the k'th row by
$A_{jk}, j > k$, and subtracting it from row $j$. Each such step requires
N-k multiplications and subtractions, for j=k+1 to N-1, or
$(N -k) \cdot (N -k- 1)$ multiplications and subtractions. This has to be
repeated for k = 0 to N-2, giving approximately $N^3/3$ multiplications and subtractions. In other words, the leading contribution to the time is $\tau N^3/3$, which is a lot better than $\tau e N!$ as with
Cramer's rule.

\TallC{When} we optimize for speed, only the innermost loop -- requiring $O (N^3 /3)$ operations needs careful tuning; the $O (N^2/2)$ 
operations -- comparing floating point numbers, dividing by the pivot, and
back-substituting -- need not be optimized bgecause for
large N they are overshadowed by the innermost loop\sepfootnote{09_09}.

% 225 
\subsection{Testing}
\TallC{Since} we have worked out a specific $3 \times 3$ set of linear equations,
we might as well use it for testing and debugging. Begin with
some test words that do the following jobs:

\begin{itemize}
  \item Create matrix and vector (inhomogeneous term)
  \item Initialize them to the values in the example
  \item Display the matrix at any stage of the calculation
  \item Display inhomogeneous term at any stage
\end{itemize}
\begin{verbatim}
\ Display words
V{ or M{{ stands torwhatanarrayputsonstack
U: G. F. x. ;
:.M (M{{--) FINIT DUP
D.LEN 0 DO CR DUP
D.LEN 0 DO DUP (--M{{ M{{ )
J}{I}} DUP>R G@ G.
LOOP
LOOP DROP ;

:.v (V{-~) DUP D.LEN
0 DO on DUP l}{0} G@ G. LOOPDROP:

\end{verbatim}
We define a word to put ASCII numbers into matrices:
\begin{verbatim}
:GEFF# BL TEXT PAD S->F;
\end{verbatim}

This word takes the next string\sepfootnote{09_10} from the input stream (set off
by ASCII blank, \textbf{BL}) and uses the HS/FORTH word \$ $\to F$ to
convert the string to floating point format and put it on the
87stack.

\textbf{GET-F\#} is used in the following:
\begin{verbatim}

: <DO-VEC> ODO GET-F# DUP I0} G!
LOOP DROP;

:TAB->VEC DUP D.LEN <DO-VEC> ;

:TAB->MAT DUP D.LEN DUP * <DO-VEC> ;
\end{verbatim}

The file EX.3 will be found in the accompanying program diskette. The explanatory notes below refer to EX.3.

\underline{Notes:}

\begin{itemize}
  \item The word .( emits everything up to a terminating) .
  \item HS/FORTH, because of its segmented dictionary, uses a word TASK to define a task-name, and FORGET-TASK to FORGET everything following the task-name. The FORTH word FORGET fails in HS/FORTH when it has to reach too deeply into the dictionary.
  \item Ex.3 uses the word \}row\{ defined below.
\end{itemize}

This is what it looks like when we load EX.3:

\begin{verbatim}
FLOAD EX.3 Loading EX.3
ReadadimenslonSvectorand 3113 matrbtlrornatablethendispiaythem
Now displaying vector.
V{ .V CR
0.0000000
4.0000000
2.0000000
Now displayhg matrbc
A{{ .M CR
1.0000000 0.000(1)") SIXJOOCDO
3.0000000 2.0000000 4.0000000
1.0000000 1.0000000 6.0000000
say Exa FORGET-TASKto graceitfly FORGET these words olt
\end{verbatim}

\subsection{Implementing pivotal elimination}
\TallC{Now} we have to define the FORTH words that will implement
this algorithm. In pseudocode, we can express pivotal elimination as shown in Fig. \ref{fig:09_01} below.

 
\begin{figure}
	\caption{
		Fig. 9-1 Pseudocode for pivotal elimination}
	\label{fig:09_01}
\end{figure}
\subsection{The program}
\TallC{Whenever} we write a complex program we are faced with the
problem "Where do we begin?" FORTH emphasizes the construction of components,
and generally cares little which component
we define first.

For example, we must swap two rows of a matrix. The most
obvious way moves data:

\begin{verbatim}
row1 -> temp
row2 -> row1
temp -> row2
\end{verbatim}

Although data moves are relatively fast, up to $N^2/2$ swaps may be
needed, each row containing N data; row-swapping is an $O(N^3)$
operation which must be avoided. Indirection, setting up a list of
row pointers and exchanging those, uses $O(N^2)$ time.

% 228

We anticipate computing the address of an element of \textbf{A\{\{} in a
\textbf{DO} loop via the phrase 

\begin{verbatim}
A{{JI}} ( Jis row,lis col)
\end{verbatim}

Suppose we have an array of row pointers and a word \verb|}row{| that
looks them up. Then the (swapped) element would be

\begin{verbatim}
A{{ J }row{ I }}.
\end{verbatim}

To implement this syntax we define\sepfootnote{09_11}

\begin{verbatim}
:swapper (n - -) l
CREATE 0 DO I , LOOP
DOES> OVER + + @ ;
\ this is a defining word
\end{verbatim}
 
We would define the array of row indices \textit{via}

\begin{verbatim}
A{{ D.LEN swapper }row{ \ create array }row{
\end{verbatim}

We have initialized the vector \verb|}row{| by filling it with integers
from 0 to N-1 while defining it.

Suppose we wanted to re-initialize the currently vectored row-index array: this is accomplished \textit{via}

\begin{verbatim}
A{{ D.LEN '}row{ refill 
\end{verbatim}

where

\begin{verbatim}
: refill (n adr --)
SWAP 0 DO I 2* DDUP + !
2 +LOOP DROP;
\end{verbatim}

To swap the two rows

\begin{verbatim}
:ADRS >R 2'OVER + SWAPR> 2' +;
(amn-a+2ma+2n)
OVAR ?SWAP \to keep track of swaps

:}SWAP (amn-)DDUP -
IF DDROP DROP \noswap - deanup

ELSE ADRS DDUP (.-121 2)
@SWAP@ (-'12[21[1I)
nor! SWAP 1

-1 ?SWAP XOR IS ?SWAP \ "' ?SWAP
THEN ;
\end{verbatim}

Test this with a 3-dimensional row-index array:

\begin{verbatim}
3 swapper }row{

:TEST ODOI}row{ CR. LOOP;
3 TEST
0
1
2 ok
\end{verbatim}

Now swap rows 1 and 2:

\begin{verbatim}
'}row{ 1 2 }SWAP 3 TEST
0
2
1 ok
\end{verbatim}
and back again:
\begin{verbatim}
'}row{ 1 2 }SWAP 3 TEST
0
1
2 ok
\end{verbatim} 

\TallC{Next}, we need a word to find the pivot element in a given
column, searching from a given n. The steps in this procedure
are shown in Fig. \ref{fig:09_02} on page \pageref{fig:09_02} below.

We anticipate using generic operations Git defined in Chapter 5 
to perform fetch, store and other useful manipulations on the
matrix elements, without specifying their type until run-time. It
is thus useful to have a place to store the type (other than the
second cell of the array data structure). We arrange this \textit{via}

% 230 Chapter 9 - Linear Algebra
\begin{figure}
\begin{verbatim}
:A{{ col# }}PIVOT

N col# 1+ DO
Yes IF
No 1 Put larger on fstack
set I.PIV = I
t__l THEN
. LOOP
; \ exit
M{{ | }{ 00"" }}I > M{{ col# }{ 00h" }}I 7 
\end{verbatim}
% Fig. 9-2 Pseudocode for finding the pivot element
    \caption{Pseudocode for finding the pivot element}
	\label{fig:09_02}
\end{figure}



\begin{verbatim}
0 VAR T		\ a place to store data type
\end{verbatim}

The rest of the definition is rendered self-explanatory by the 
vertical commenting style (a must for long\sepfootnote{09_12} words):
\begin{verbatim}
OVAR Length \Iength of matrix
OVAR COL \currentcol#
OVAR I.PIV
OVAR a{{
: INITIALIZE (M{{ -- ::--)
IS a{{
a type@ IS T
a LEN@ IS Length ;

\ |.PIV is used to return the result
\ a{{ stores the adress of M{{

\ initialize T

\ length - > Length


2))PIVOT (M{{ col-- ::--)

IS COL ZECOL IS LPN

INITIAU

fl: COL row COL}} (--aog.ofl[M{% oot,ool}}]t)

> 8 SF >F (87:--|1st.olt)

R> COL1+ \Sloop|in£gLCOL+1

DO I .
a{{ I }row{ COL }} (- - seg.ofl[M{flI,col}}] t)
>FS GABS FS>F(87: - |old.olt Ineweltl)
xoup F< \test' newett > olden
IF FSWAP IISI.PIV THEN FDROP

LOOP \ondloop

FDROP ; \deanupfstack

\ Usage: A{{ 2 "PIVOT
\end{verbatim}

\underline{Notes:}

\begin{itemize}
  \item We avoid filling up the parameter stack with addresses that have to be
	  juggled, by putting arguments in named storage locations (\textbf{VAR}s).
		We anticifiate setting all the parameters in the beginning using
		\textbf{INITIALIZE} which can be extended later if necessary.
  \item We have used a word from the COMPLEX arithmetic lexicon, namely
	  \textbf{XDUP( FOVER FOVER)} to double-copy the contents of the fstack,
		since the test $F<$ drops both arguments and leaves a flag.
  \item There is little to be gained by optimizing (and if we did we should
	  have had to avoid the generic \textbf{Gx} words because they cannot be optimized
		by the HS/FORTH recursive-descent optimizer) because only $N^2/2$ time
		is used, negligible (for large N) compared with the $N^3/3$ in the
		innermost loop.
\end{itemize}



The \textbf{Gx} operations are found in the file \textbf{GLIB.FTH}\,.

To test the word \verb|}}PIVOT| we can add some lines to the test file:

\begin{verbatim}
G! .(A{{O})PIVOT mv .) BL EMIT A{{0)}PNOT IPN.
on .(A{{1})PNOT IPIV .) BL em A{{1}}PNOT IPN .
on .(A{{2})PIVOT IPN .) BL am A({2}}PNOT new.
\end{verbatim}
The result is the additional screen output


\begin{verbatim}
A{{ o }}PIVOT IPIV 1
A{{1}}P|VOT IPIV 1
A{{ 2 }}PIVOT IPIV 2
OK
\end{verbatim}

From our pseudocode expression (see Fig. \ref{fig:09_01}, p. \pageref{fig:09_01}) of the
pivotal elimination algorithm we see that the next operation is to
multiply a row by a constant. To do this we define

\begin{verbatim}
OVAR ISTEP
\ include phrase

T #BYTES DROP IS ISTEP
\ in INITIALIZE

: DO(ROW*X)LOOP (seg off Lfin I.beg - - :: x - - x)
DO DDUP I + DDUP T >FS G'NP TFS>
ISTEP /LOOP DDROP ;

\ G*NP means "(generic) multiply, no pop"

:}}ROW*X M{{ row - :: x - x)
UNDER }row{ 0 A (- - r seg offt)
DROP ROT (- -seg off r)
ISTEP * Length ISTEP * SWAP \Ioop indices
DO(ROW*X)LOOP ; \Ioop

\Ex: A{{ 2 }ROW*X

\end{verbatim}

\begin{itemize}
	\item  While \regc{DO(ROW*X)LOOP} and \regc{\}\}ROW*X} clear the parameter stack in approved FORTH fashion, they leave the constant
		multiplier \textbf{x} on the fstack. That is, we anticipate next multiplying the corresponding row of the inhomogeneous term \textbf{b} (in
		the matrix equation \textbf{A-x = b}) by the same constant \textbf{x}.
	\item We assume \regc{INITIALIZE} (including \regc{INIT.ISTEP}) has been
		invoked before \verb|}}PIVOT| or \verb|}}ROW*X|.
  \item Anticipating the (possible) need to optimize, we factor the loop
right out of the word. We also com me the addresses fast by
putting base addresses on the stack and then incrementing
them by the cell-size, in bytes.
\end{itemize}

% aims-mum 233

Subtracting row I (times a constant) from row J is quite similar to
multiplying a row by a constant. The process can be broken down
into the pseudocoded actions in Fig.\ref{fig:09_03} below.

 
\begin{figure}
\begin{verbatim}

 

® no (K-JTOL)
x e@

M[I,K] e@ G'NP (::- - x M[l,K]'X)
M{J,K] 6@ GR- (::--XM[J,K]-X'M[I,K])
M{J.K] Gl

LOOP

 

\end{verbatim} 
	\caption{Fig. "Steps In \regc{\}\}R1-R2*X}}
	\label{fig:09_03}
\end{figure}

A first attempt might look as follows:

\begin{verbatim}
\auxiliarywords

:4DUP DOVER DOVER ;

(abcd--abcdabcd)

: + + (51.0152.02n -  1.01 +n 52.02+n)
DUP>R + ROT R> + -ROT;

:}}R1-R2"X (r1r2-- ::x--x)

> R > R

a R> row 0 DROP \M r10
aila@imioli one.» \Miiaoli
Length iSTEP " \ L'lSTEP (upper limit)

R> ISTEP * \ r2'ISTEP (lower limit)

DO 4DUP I ++ \duplicateaincbaseadrs

T >FS G'NP (2: - xx*M[r2,k])
DDUP T >FS GR-
T >FS \ I result
ISTEP [LOOP DDROP DDROP ;
\ INITIALIZE is assumed

\end{verbatim} 
As with \verb|}}ROW*X| we avoid the execution-speed penalty of
calculating addresses within the loop by not using the phrase
\begin{verbatim}
a{{ I }row{ J }} .
\end{verbatim}
However, we are doing a lot of unnecessary work inside the loop
by making 5 choices based on datatype. There are also a lot of
unnecessary moves to/from the ifstack. This is an example where
speed has been sacrificed to compactness of code: one program
solves equations of any ("scientific") data format - REAL*4,
REAL*8, COMPLEX*8 and COMPLEX*16. 

Conversely, we can both eliminate the extra work and accelerate
execution simply by defining a \textit{separate} inner loop for each data;
type, letting the choice take place once, outside the inner loop.
This multiplies the needed code fourfold (or more-fold, if we
optimize).

The 4 inner loops are
\begin{verbatim}

: Re.LP (\S1.01 52.02 L'istep r2*istep - - :2 x - - x)
FS>F >R
DO 4DUP I + + R32@L F*NP
DDUP R32@L FR- R32IL
4 /LOOP R> F>FS ;

: DRe.LP ($1.01 $2.02 L'istep r2'istep - - z: x - - x)
FS>F >R
DO 4DUP l + + R64@L F*NP
DDUP R64@L FR- R64!L
8 /LOOP R> F>FS ;

: X.LP (51.01 52.02 L*istep r2*istep - - z: x - - x)
FS>F >R
DO 4DUP l + + CP@L X'NP
DDUP CP@L CPR- CPIL
8 /LOOP R> F>FS ;

: DX.LP (\S1.01 52.02 L'istep r2'istep - - :: x - - x)
FS>F >R
DO 4DUP I + + DCP@L X'NP
DDUP DCP@L CPR- DCPIL
16 /LOOP R> F>FS;
\end{verbatim} 

Not surprisingly, the loops contain some duplicate code, but this
is a small price to pay for the speed increase. A significant further
increase can be obtained easily, using the HS/FORTH recursive-
descent optimizer to define these inner loops, or by redefining
the loops in assembler (for ultimate speed).

Now we can redefine \verb|}}R1-R2*X| using vectored execution, \textit{via}
HS/FORTH's \regc{CASE: ... ;CASE} construct, or the equivalent high-
level version given here:

\begin{verbatim}
: CASE: CREATE
DOES> OVER + + @ EXECUTE ;
: ;CASE [COMPILE] ; ; IMMEDIATE

CASE: DO(R1-R2"X)LOOP
Re.LP DRe.LP X.LP DX.LP ;CASE

2}}R1-R2'X (r1 r2-- ::x--x)

>R >R

2112a mm; 328: {main
Length I.STEP * \ ULSTEP (upper limit)
R> I.STEP ' \ r2'I.STEP (lower limit)

T DO(R1-R2'X)LOOP DDROP DDROP ;
\ We assume INITIALIZE has been invoked
\end{verbatim} 

Another word is needed to perform the same manipulation on
the inhomogeneous term. Since this latter process runs in $O(N^2)$
time we need not concern ourselves unduly with efficiency.
\begin{verbatim}

:}V1-V2*X (r1 r2-- ::x--)
SWAP >R >R
b{ 0.0 } DROP DDUP \V[0] V[O]
R> irowilSTEP' + >FS G'
R> row ISTEP " + DDUP >FS GR- FS> ;

\end{verbatim} 
\underline{Notes:}

\begin{itemize}
	\item  After \regc{\}V1-V2*X} executes, the multiplier x is no longer needed,
		so we drop it here by using \regc{G*} rather than \regc{G*NP}.
\end{itemize}

We now combine the words \regc{\}SWAP}, \regc{\}\}PIVOT}, \regc{\}\}ROW*X}.
\}R1-R2*X and \}V1-V2*X to implement the triangularization portion of pivotal elimination.

% 236
Since the Gaussian elimination method makes it very easy to
compute the determinant of the matrix as we go, we might as well
do so, especially as it is only an O(N) process. By evaluating the
detemiinants of simple cases, we realize the determinant is simp-
ly the product of all the pivot elements, multiplied by
$(-1)^{\#swaps}$. Hence we need to keep track of swaps, as well as to
multiply the determinant (at a given stage) by the next pivot
element. The need to do these things has been anticipated in
defining \regc{\}SWAP} above.

Computing the determinant also lets us test as we go, that the
equations can be solved: if at any stage the determinant is zero,
(at least) two of the equations must have been equivalent. Should
it be necessary to test the condition\sepfootnote{09_13} of the equations, this too
can be found as we proceed, by computing the determinant.

Here is a simple recipe for computing the determinant, with
checking to be sure it does not vanish identically. We use the
intelligent fstack (ifstack) defined in Ch. 7.

\begin{verbatim}
DCOMPLEX SCALAR DET \ room for any type
:INIT\_DEr T 'DEI' I \set Type
TG=1 DEI' GI; \setdet=

%1.E-10 FCONSTANT CONDITION

:DETERMINANT (- - 2: x - - x)
DET >FS G*NP
?SWAP IF GNEGATE THEN
FS.DUP GABS FS>F CONDITION F<
ABOR'l" determinant too small!' DET FS> ;

\end{verbatim} 

% 237

Now we define the word that triangularizes the matrix:
\begin{verbatim}
0VARb( \tokeepthesteokehort
: INITIAUZE (M{(V{o- ::--)

le

lSa

3 .TYPE IS T

a D.LEN is Length

a D.TYPE #BYTES DROP ISISTEP

iN .DET ; \setdet=1

:}/PIVOT }row{) DROP DDUP T >FS G' TFS>;
(segofIt--::x--)

:TRIANGULAFIIZE (M{{ V{ - -)

INITIAUZE
Length 0 DO \ loop 1 -- by rows
a{{l} PIVOT \findpivotincoil
' )row I l.PlV }SWAP \ exchange rows
a{{ I }row{ I }} >FS \pivot->iistack
DETERMINANT \ mic det
1/6 a{{ i }}ROW'X \row i Ipivot
b{ l }/PIVOT \inhom. term /pivot
Length l1+ DO \ioop2-byrows
a l}row{J}} >FS \x->ifstack
a lJ}}R1-R2"X

\ row[i] = row[i]-row[j]'x
b{ I J }V1-V2*X
\ same for b{ and drop x
LOOP \ and loop 2
LOOP ; \end loop 1
\Usege: A{{ B{ TRIANGULARIZE
\end{verbatim}

Now at last we can back-solve the triangularized equations to find
the unknown x's. The word for this is \verb|}BACK-SOLVE|, defined
as follows:

% 238

% Chapter 9 - Linear Algebra Scientific FORTH
\begin{verbatim}
: }BACK-SOLVE (- - )
0 LENGTH 2- DO \ outer loop
T G=0
LENGTH I 1 + DO \ inner loop
a? J }row{ I }} > PS
b I }row{ } >FS G*G+\
LOOP \ inner loop
b{ l ){ } DROP DDUP T >FS \
GR- T FS > \
-1 +LOOP ; \outer loop J
\end{verbatim} 

Putting the entire program together we have the linear equation
solver given in the file SOLVE1. Examples of solving dense $3\times 3$
and $4\times 4$ systems are included for testing purposes.

\subsection{Timing}
\TallC{We} should like to know how much time it will take to solve a 
given system. (Of course it is also useful to know whether the
solution is correct!) We time the solution of 4 sets of N equations,
with 4 different values of N. The running time can be expressed
as a cubic polynomial in N with undetermined coefficients:
% ( restart from 19)
\addtocounter{equation}{-3} % the next equation is back with 3 22 -> 19
\begin{equation}
	T_N=a_0+a_1N^1+a_2N^2+a_3N^3 \label{eq:09_19_}
\end{equation}

Evaluating \ref{eq:09_19_} for 4 different values of N, and measuring the 4
times $T_{N_i}$ , we produce 4 inhomogeneous linear equations in 4 
unknowns: $a_i , i = 0, 1, 2, 3$. As luck would have it, we just happen to have on hand a linear equation solver, and are thus in a
position to determine these coefficients numerically.

For this timing and testing chore we need an exactly soluble dense
system of linear equations, of arbitrary order. The simplest such
system involves a rank-1 matrix, that is, a matrix of the form \sepfootnote{09_14}:
%  (20_)
\begin{subequations}
\begin{equation}
 \label{eq:09_20_}
A = I - u v^\dagger
\end{equation}
In terms of matrix elements,

% (20'_)
\begin{equation}
 \label{eq:09_20prime_}
	A_{ij} = \delta_{ij} - u_i v_j^*
\end{equation}
\end{subequations}

The solution of the system $A\cdot x = b$ is simple: using the standard
notation
% (21_)
\begin{equation}
 \label{eq:09_21_}
	v^\dagger \cdot x \equiv (v,x) = \sum_{i}v_i^*x_i
\end{equation}
we have
\begin{equation}
 \label{eq:09_22_}
	x_i = b_i + (v,x)\,u_i
\end{equation}

The coefficient $(v, x)$ is just a (generally complex) number; we
compute it from \ref{eq:09_22_}\textit{via}
\begin{equation}
 \label{eq:09_23_}
	(v,x)=(v,b)+(v,u)(v,x)
\end{equation}
or
\begin{equation}
 \label{eq:09_24_}
	(v,x)=\frac{(v,b)}{1-(v,u)}\quad .
\end{equation}
% (25)
Substituting \ref{eq:09_24_}back in\ref{eq:09_22_}, we have
\begin{equation}
 \label{eq:09_25_}
	x_i=\frac{(v,b)}{1-(v,u)}
\end{equation}
Everything in \ref{eq:09_25_} is determined, hence the solution is known in
closed form.

An example that embodies the above idea is given in the file
EX.20, where we make the special choices

\begin{equation*}
 \label{eq:09_25_}
	\mathbf{u}=
	\begin{pmatrix}
		1\\ 
		1\\ 
		1\\ 
		\dots\\ 
		\dots\\ 
		\dots
	\end{pmatrix}
	\, ,\, 
	v=\frac{1}{2N} u\, , \, 
	\mathbf{b} =
	\begin{pmatrix}
		1\\ 
		0\\ 
		1\\ 
		0\\ 
		\dots\\ 
		\dots
	\end{pmatrix}
	,\, \,  
	\mathbf{x}=
	\begin{pmatrix}
		1.5\\ 
		0.5\\ 
		1.5\\ 
		0.5\\ 
		\dots\\ 
		\dots
	\end{pmatrix}
	\, , N \; \textrm{even}
\end{equation*}
We give the times only for the case of a highly optimized inner
loop (written in assembler), for the type REAL*4:

% Table 9-1
\begin{table}
	\centering
	\caption{Execution times for various N (9.54 MHz 8086 + 8087 machine)}
	    \bigskip
    \label{tbl:09_01} 
	\setlength{\tabcolsep}{40pt}
        \begin{tabular}{|ll|}
            \hline & \\
			N   &  Time\\
			& \\
			20  &  1.20\\
			& \\
			50  &  7.75\\
			& \\
			75  &  19.6\\
			& \\
			100 &  38.8\\
            & \\
            \hline
        \end{tabular}
\end{table}

The coefficients of the cubic polynomial extracted from the above
are (in seconds, 9.54 MHz 8086 machine, machine-coded inner loop)

\begin{equation*}
	\begin{split}
		a_0 &= 0.32727304\\
		a_1 &= -0.01084850\\
		a_2 &= 0.00241636\\
		a_3 &= 0.00001539
	\end{split}
\end{equation*}

from them we can extrapolate the time to solve a $350 \times 350$ real
system to be about 16 minutes. On a 25 MHz 80386/80387 machine with 32-bit addrgsing implemented, the time should
decrease five- to tenfold \sepfootnote{09_15}.

It is interesting to explore a bit further the extracted value of $a_3$,
which is $\frac{1}{3}$ the time needed to evaluate the inner loop (defined
above as Re.LP and hand-coded in assembler for ultimate
speed). Converting to clock cycles on an IBM-PC compatible
(running at 9.54 MHz) we have an average of 440 clock cycles per
traversal (of this loop). Recall the operations that are needed: a
32-bit memory fetch, a floating-point multiply, another 32-bit
memory fetch, a floating-point subtraction, and a 32-bit store.
The initial fetch-and-multiply can be be compressed into a single
co-processor operation, as can the fetch-and-subtract. The times
in cycles for these basic operations are\sepfootnote{09_16}
% Table 9-2
\begin{table}
	\centering
	\caption{Execution times of innermost loop operations}
	    \bigskip
    \label{tbl:09_02} 
	\setlength{\tabcolsep}{10pt}
        \begin{tabular}{|ll|}
			\hline &  \\
			\underline{Operation} & \underline{Time} \; (cpu clocks)\\
			&  \\
			memory FMUL &  133  \\
			&  \\
			memory FSUBR &  128  \\
			&  \\
			memory FSTP &  100  \\ 
			&  \\
			overhead &  61  \\ 
			&  \\
            \hline &  \\
			Total &  422  \\ 
			&  \\
            \hline
        \end{tabular}
\end{table}
 
% 242

\section{Matrix Inversion}
\TallC{We} see from Table \ref{tbl:09_02} above that the time computed from the
cpu and coprocessor specifications (422 clocks) is close to the
measured time (440 clocks). The slight difference doubtless
comes both from measurement error and from the fact the
timing of CISC chips is not an exact art (for example, there a \underline{??}
periodic interruptions for dynamic memory refreshment and f \underline{??}
the system clock).

Finally, we confirm that little is to be gained by optimizing out
loops. Suppose, \textit{e.g.}, we could halve $a_2$ by clever programming
then we should cut the N = 350 time from 16 to 13 minutes, and 
the time for N = 1000 by some 20 minutes in 5 hours, or 6\%.

We introduce this subject with a brief discourse on liners
algebra of square matrices.

\subsection{Linear transformations}
Suppose A is a square $(N\times N)$ matrix and x is an N-dimensional
vector (a column, or $(N\times 1)$ matrix). We can think of the symbolic
operation
 % (26)
\begin{equation}
	\label{eq:09_26}
	y= A -x
\end{equation} 
as a linear transformation of the column x to a new column y. In
terms of matrix elements and components,

\begin{equation}
	\label{eq:09_27}
	y_m = \sum_{n=0}^{N-1} A_{mn} x_n
\end{equation}

The transformation is \textbf{linear} because if x is the sum of 2 columns
(added component by component)
\begin{equation}
	\label{eq:09_28}
	x=x^{(1)}+x^{(2)}\;,
\end{equation}
we can calculate A 1 either by first adding the two vectors ant:
then transforming, written
\begin{equation}
	\label{eq:09_29}
	A \cdot x = A \cdot \Big( x^{(1)}+x^{(2)} \Big)
\end{equation}
or we could transform first and then add the transformed vectors.
The identity of the results is called the \textbf{distributive law}
\begin{equation}
	\label{eq:09_30}
	A \cdot x = A \cdot \Big( x^{(1)}+x^{(2)} \Big) \equiv   A \cdot x^{(1)}+ A \cdot x^{(2)} 
\end{equation}
of linear transformations.

\subsection{Matrix multiplication}

Now, suppose we had several square matrices, \textbf{A, B, C,...} We
could imagine performing successive linear transformations on a
vector x \textit{via}
\begin{subequations}
	\begin{equation}
		\label{eq:09_31a}
		y= A \cdot x% (31a)
	\end{equation}
	\begin{equation}
		\label{eq:09_31b}
		z= B \cdot y % (31b)
	\end{equation}
	\begin{equation}
		\label{eq:09_31c}
		w = C \cdot z  % (31c)
	\end{equation}
\end{subequations}

These can conveniently be written

\begin{equation}
	\label{eq:09_32}
	w=C \cdot \Big( B \cdot (A \cdot x) \Big) \; .%  (32)
\end{equation}

The concept of successive transformations leads to the idea of
multiplying two matrices to obtain a third:
% (33)
\begin{equation}
	\label{eq:09_33}
	D = B \cdot A \quad E = C \cdot D
\end{equation}
D=B-A E=C-D (33)
In terms of matrix elements we have, for example
\begin{equation}
	\label{eq:09_34}
	D_{ik} = \sum_{j=0}^{N-1} B_{ij} A_{jk}
\end{equation}

The important point is that the (matrix) multiplications may be
performed in any order, so long as the left-to-right ordering of
the factors is maintained:
\begin{equation}
	\label{eq:09_35}
	C \cdot \big( B \cdot A \big) \equiv \big( C \cdot B \big) \cdot A
\end{equation}
Equation \ref{eq:09_35} is known as the associative law of matrix
multiplication. Finally, we note that -- as hinted above -- the left-to-right
order of factors in matrix multiplication is significant. That is, i
general,
\begin{equation}
	\label{eq:09_36}
	A \cdot B \neq B \cdot A
\end{equation}
We say that, unlike with ordinary or even complex arithmetics
matrix multiplication -- even of s uare matrices -- does not in
general obey the commutative law \sepfootnote{09_17}.

\subsection{Matrix Inversion}
With this introduction, what does it mean to invert a matrix? First
of all, the concept can apply only to a square matrix. Given an
$N \times N$ matrix $A$, we seek another $N \times N$ matrix $A^{-1}$ with the property that
\begin{equation}
	\label{eq:09_37}
	A^{-1} \cdot A \equiv A \cdot A^{-1}
\end{equation} 
where I is the unit matrix defined in the beginning of this chapter
(Eq. 4  \underline{remark : probably it is 3}-- 1's on the main diagonal, 0's everywhere else). We have
implied in Eq. \ref{eq:09_37} that a matrix that is an inverse with respect to
left-multiplication is also an inverse with respect to right-multiplication. Put another way, we imply that
\begin{equation}
	\label{eq:09_38}
	\Big(A^{-1} \Big)^{-1} \equiv A
\end{equation} 
The condition that -- given A -- we can construct $A^{-1}$ is the same
as the condition that we should be able to solve the linear equation
\begin{equation*}
	A \cdot x = b \,;
\end{equation*} 
 namely, $\det(A) \neq 0$.

\subsection{Why invert matrices, anyway?}

We have developed a linear equation solver program already
-- why should we be interested in a matrix inversion program?

Here is why: The time needed to solve a single system (that is,
with a given inhomogeneous term) of linear equations is conditioned by the number of floating-point multiplications required:
about $N^3/3$. The number needed to invert the matrix is about $N^3$,
roughly $3\times$ as many, meaning roughly $3\times$ as long to invert as to
solve, if the matrix is large. Clearly there is no advantage to
inverting unless we want to solve a number of equations with the
same coefficient matrix but with different inhomogeneous terms.
In this case, we can write
\begin{equation}
	\label{eq:09_39}
	x = A^{-1} \cdot b
\end{equation} 
and just recalculate for each \textbf{b}. Clearly this breaks even - relative
to solving 3 sets of equations - for 3 different \textbf{b}'s and is superior
to re-solving for more than 3 \textbf{b}'s.

s55 Anmmpie

Let us now calculate the inverse of our 3x3 matrix from before.
The equation is (let c = A " be the inverse")

105 000301 can 100
3 2 4 C10 C" C12 = 0 1 0 (40)
116 CanCa 001

 

18. Note:ifCisaright-inverse,AC-l,itkalsoa|dl-inverse,CA=l.

OJUIUIVNanim-Alrldasraaarvad.

246

CMptarO-Urngebra SclentlflcFORT

It is easy to see that Eq. 40 is like 3 linear equations, the unkn
being each column of the matrix C. The brute-force method I
calculate A is then to work on the right-hand-side all at once
we triangularize, and back-solve. It is easy to see that
gularization by pivotal elimination leads to

1 2/3 Vs can cat can 0 V3 0
0 1 *'1/2 C10 Ctr C12 = '3': V2 0 (41)
0 O 1 C20 C12 022 V13 "V13  13 l

results of back-solving, column by column. This is N times a.

if

1
total time required for brute-force inversion is =5 gN3. By keep-

To construct the inverse we replace the right-hand side with thq

much work as back-solving a single set of equations, hence th

ing track of zero elements we could reduce this time by anothel
%N3, thereby obtaining the theoretical minimum (for Gaussian
elimination) of 0(N3). However, the brute-force method is un-
satisfactory for another reason: it takes twice the storage of more
sophisticated algorithms. Modern matrix packages therefore "8:"
LU decomposition for both linear equations and matrix inversion.

 

\S4 LU decomposition

We now investigate the LU decomposition algorithm". Sup-
pose a given matrix A could be rewritten

awam 1m 0 0\}lmflm
A: 310311 = L'U = 1401110 0  11 (42)
.................. 0 0

then the solution of

 

19.

Sec, 2.3., WH. Press, B.P. F'Iannery, SA. Teukolsky and W.T. Vetterling. Numerical Recipe:
(Cambridge University Press, Cambridge, 1986), p. 31H.

owe-mm 241

(L-u)-nL-(u-x) -b (43)
can be found in two steps: First. solve
Lty- b (44)
for
y: U-x (45)
via
looYo = bo
MOYO'l'ltIYI =01 (46)
130\% + 421Y1+ la)': = D:
\textit{etc.}

which can be solved successively by forward substtution. Next
solve 45 successively (by back-substitution) for x:

l'N-tN-t XN-1 = YN-t
I'M-2 N-2 XN-z + I'M-2 N-1 xN-i = YN-z (47)
\textit{etc.}

The n'th term of Eq. 46 requires n multiplications and n additions.
Since we must sum n from 0 to N-l, we find N(N-1)/2 =4 Nz/Z
multiplications and additions to solve all of 46. Similarly, solving

47 requires about N2/2 additions and multiplications. Thus, the
dominant time in solving must be the time to decompose accord-

ing to Eq. 42.

The decomposition time is =5 N 3/3, and the method for decom-
posing is described clearly in Numerical Recipes. The equations
to be solved are

N-l

[a luk/'h =Araa (48)

constituting N2 + N equations for N2 unknowns. Thus we may
arbitrarily choose I... = 1.

eJuunMNobmm-nmm.

Chapters-UnearNgobra ScbndflcFORTM

The equations 48 are easy to solve if we do so in a sensible ordefi
Clearly,

1mk50,m>k il
(49)
"knaos k<n

so we can divide up the work as follows: for each 11, write

 

 

m-l
.umn =Amn - kzo Amkflkn ' m = 0,1,...,'l

(50)

n-l
1m =i(A..... - 2 Amp") , m =n+1,n+2,...,N-1 -.
.unn i=0

 

Inspection of Eq. 50 makes clear that the terms on the right side!
are always computed before they are needed. We can store tln
computed elements A." and uh, in place of the corresponding
elements of the original matrix (on the diagonals we store  4..

since A", = 1 is known).

To limit roundoff error we again pivot, which amounts to permutl
ing so the row with the largest diagonal element is the currenf
one. Much of the code developed for the Gauss eliminatios
method is applicable, as the file LU.FI'H shows.

