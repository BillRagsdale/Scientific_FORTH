% All footnotes are here.

% Chapter 1 footnotes
\sepfootnotecontent{01_01}{This description refers to FORTH's compilation scheme. See, e.g. R.G. Loeliger, \textit{Threaded Interpretive Languages} (Byte Publications, Inc., Peterborough, NH, 1981). We shall have more to say about it in Chatper 2.}

\sepfootnotecontent{01_02}{Although some interpreted FORTRANs such as WATFOR have been developed.}

\sepfootnotecontent{01_03}{The original version of FORTRAN included naming conventions such that names beginning with letters I, J, K, L, M, and N are assumed to be integers, while those beginning with other letters are assumed to be single-precision floating point numbers. Subsequent versions have maintained this convention for backward compatibility.}

\sepfootnotecontent{01_04}{The items between parentheses, (\dots), and following a backslash, "\textbackslash", are comments.}

\sepfootnotecontent{01_05}{A \textbf{stack} is a data structure like a pile of cards, each containing a number. New numbers are added by placing them atop the pile, numbers are also deleted from the top. In essence, a stack is a "last-in, first-out" buffer.}

\sepfootnotecontent{01_06}{L. Brodie, \textit{Thinking Forth} (Prentice-Hall, Inc., Englewood Cliffs, New Jersey, 1984). M. Ham, "Structured Programming", \textit{Dr. Dobb's Journal}, July 1986.}

\sepfootnotecontent{02_01}{L. Brodie, \textit{Starting FORTH}, 2nd ed. (Prentice-Hall, NJ, 1986), referred to hereafter as \SF.}

\sepfootnotecontent{02_02}{L. Brodie, \textit{Thinking FORTH} (Prentice-Hall, NJ 1984), referred to hereafter as \TF.}

\sepfootnotecontent{02_03}{M. Kelly and N. Spies, FORTH: a Tea and Reference (Prentice-Hall, NJ , 1986), referred to hereafter as \FTR.}

\sepfootnotecontent{02_04}{Successive words in the input stream are separated from each other by blank spaces, ASCII 20hex, the standard FORTH delimiter.}

\sepfootnotecontent{02_05}{We will explain about the stack in 2.3.}

\sepfootnotecontent{02_06}{Since FORTH uses words, when we enter an input line we say the corresponding phrase.}

\sepfootnotecontent{02_07}{This level could be either the outer interpreter or a word that invokes \bc{NEW-WORD}.}

\sepfootnotecontent{02_08}{defined as \bc{: -ROT ROT ROT ;}}